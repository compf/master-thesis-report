\begin{comment}
\section{Implementierte Metriken}\label{chapter:metrics}
In diesem Abschnitt wird ein Überblick über einige Metriken gegeben, die in der finalen Version des Tools implementiert sind. Dabei werden die Hintergründe der Metriken erläutert und Vor- und Nachteile der einzelnen Metriken genannt. Die implementierten Metriken lassen sich -- abhängig von den berücksichtigten Aspekte der Dokumentation -- grob in drei Kategorien einteilen.

\subsection{Metriken, welche die Abdeckung überprüfen}\label{chapter:metrics_coverage}
Eine grundsätzliche Möglichkeit zur Bewertung der Dokumentationsqualität ist es, die Abdeckung der Dokumentation zu prüfen. Dabei werden aus einer  Teilmenge aller Komponenten alle dokumentierten Komponenten mit dem maximalen Wert 100 bewertet und alle undokumentierten Komponenten mit 0 bewertet.  Diese Metriken werden in der wissenschaftlichen Literatur auch als \textit{ANYJ} oder \textit{DIR} bezeichnet \cite[S.~5]{HowDocumentationEvolvesoverTime}.

Bei der Wahl der Teilmenge gibt es zwei naheliegenden Möglichkeiten. Bei der ersten Möglichkeit werden alle Komponenten überprüft, sodass die Teilmenge gleich der Gesamtmenge der Komponenten ist. Für eine gute Bewertung wird somit verlangt, dass möglichst jede Komponente dokumentiert wird. Bei der zweiten Möglichkeit werden nur die öffentlichen Komponenten untersucht und alle nicht öffentlichen Komponenten so behandelt, als ob sie nicht existieren würden. Dies hat den Vorteil, dass nur Komponenten untersucht werden, die wahrscheinlich von anderen Komponenten verwendet werden und somit besser verstanden werden müssen \cite[S.~253]{JavadocViolationsandTheirEvolutioninOpen-SourceSoftware}. 


Durch diese Metriken kann geprüft werden, ob ausreichend viele Komponenten dokumentiert sind. Allerdings überprüfen diese Metriken nur, ob die Komponente dokumentiert ist oder nicht. Ein leerer, sinnloser oder sachfremder Kommentar würde dennoch mit 100 Punkten bewertet werden. Falls alle Komponenten untersucht werden, kann außerdem das Problem auftreten, dass einige Komponenten nur in einem kleinen Bereich des Programms verwendet werden, sodass sie für die Dokumentationsqualität weniger relevant sein können. 

Als Erweiterung der Abdeckungsmetriken kann bei Methoden überprüft werden, ob auch ihre Parameter und der Rückgabewert dokumentiert sind. Dies ist sinnvoll, da diese auch Teil der Methode sind und zum Verständnis der Methode beitragen \cite[S.~5]{HowDocumentationEvolvesoverTime}. Zudem kann es sinnvoll sein, triviale Getter und Setter, die den Wert eines privaten Feldes auslesen oder verändern, bei der Bewertung auszuschließen. Schließlich haben diese Methoden einen klaren Zweck, der oft keiner weiteren Kommentierung bedarf \cite[S.~254]{JavadocViolationsandTheirEvolutioninOpen-SourceSoftware}.

Bei der Abdeckung der Dokumentation von Methoden könnte auch die Länge der Methode berücksichtigt werden, was aber in der Literatur nicht erwähnt wird. Dabei wird jede Methode mit ihrem \ac{LOC} gewichtet und die Summe der \ac{LOC} der dokumentierten Methoden innerhalb einer Klasse wird durch die Summe der \ac{LOC} aller Methoden der Klasse geteilt. Dies hat den Vorteil, dass undokumentierte Getter und Setter, die nur wenige Codezeilen haben, das Ergebnis nicht so stark beeinflussen, aber trotzdem berücksichtigt werden. 


\subsection{Metriken, welche die Semantik überprüfen}\label{chapter:metrics_semantic}

Da nicht nur das Vorhandensein, sondern auch die Aussagekraft und Verständlichkeit der Dokumentation wichtig sind, bietet das Tool weitere Metriken an. Diese Metriken überprüfen die Semantik, also den Inhalt, der Dokumentation und können so Aufschluss darüber geben, ob die Dokumentation tatsächlich hilfreich ist. 

Zur Bewertung der Verständlichkeit eines englischsprachigen Textes ist der \textbf{Flesch-Score} sehr bekannt \cite[S.~21]{ThePrinciplesofReadability}. Dies ist eine Formel zur heuristischen Bewertung der Lesbarkeit eines Textes, welches die Anzahl der Sätze, Silben und Wörter berücksichtigt. Die Formel lautet:
\begin{equation}
   206,835-1,015*\frac{W}{S}-84,6*\frac{H}{W}
\end{equation}

Dabei ist $S$ die Anzahl der Sätze, $W$ die Anzahl der Wörter und $H$ die Anzahl der Silben. Die Formel liefert einen Wert von 0 bis 100 zurück, wobei 0 auf einen sehr komplizierten und 100 auf einen leichten Text hindeutet. Diese Formel und verwandte Formeln werden auch von verschiedenen US-Behörden verwendet, um die Lesbarkeit ihrer Dokumente zu verbessern \cite[S.~72]{AutomaticQualityAssessmentofSourceCodeComments:TheJavadocMiner}. Die Implementation der Metrik nimmt an, dass ein Flesch-Score von 70 mit 100 Punkten bewertet werden sollte, da mehr als 80~\% der US-Bevölkerung solche Texte verstehen können und diese Texte weder zu leicht noch zu schwer sein sollen \cite[S.~22]{ThePrinciplesofReadability}. Für kleinere Flesch-Score werden weniger Punkte vergeben. Für größere Flesch-Scores werden ebenfalls Punkte abgezogen, aber nicht mehr als 15, da sehr leichte Texte besser sind als schwierige Texte. 
Bei der Verwendung des Flesch-Scores sollte beachtet werden, dass die Bestimmung der Silbenzahl eines Wortes nicht immer trivial ist. Bei der Entwicklung des Tools musste bei einem Austausch einer Bibliothek die Testmethode angepasst werden, da das Wort \enquote{themselves} plötzlich drei statt zwei Silben hatte. Dies hängt natürlich auch von der Aussprache und somit von kulturellen Gegebenheiten ab. Dies gilt natürlich auch für die Bestimmung von Wörtern und Sätze. 

Der \textbf{Gunning-Fog-Index} ist eine ähnliche Formel. Sie berücksichtigt nicht die Anzahl der Silben, sondern die Zahl der komplizierten Wörter. Ein Wort gilt dabei als kompliziert, wenn es mehr als zwei Silben hat \cite[S. 24]{ThePrinciplesofReadability}. Dabei gibt der Gunning-Fog-Index die Anzahl an Schuljahren zurück, die ein Leser absolviert haben muss, um den Text gut verstehen zu können. Die Definition des komplizierten Wortes ist aber umstritten. So ist beispielsweise das Wort \enquote{vacation} drei Silben lang, aber nicht unbedingt kompliziert \cite[S.~10]{bogert1985defense}.

\bigskip
Eine weitere Möglichkeit zu Bewertung der Semantik der Dokumentation ist ein Vergleich der Dokumentation mit dem Namen der dokumentierten Komponente. Ein Kommentar, der einen Großteil des Namens der Komponente wiederholt, bietet keinen Mehrwert. Auch ein Kommentar, der keinen Zusammenhang mit der dokumentierten Komponente erkennen lässt, verliert an Nutzen \cite[S.~86]{Qualityanalysisofsourcecodecomments}.
Um diese \textbf{Kohärenz} zu messen, kann die Anzahl der gemeinsamen Wörter zwischen Dokumentation und des Namens der dokumentierten Komponente ermittelt werden und dies durch die Zahl der Wörter der Dokumentation geteilt werden. Die Autoren in \cite[S.~87]{Qualityanalysisofsourcecodecomments} vertreten die Ansicht, dass dieser \textbf{Kohärenzkoeffizient} von mehr als 0,5 oder gleich 0 auf eine schlechte Dokumentation hindeutet, da es im ersteren Fall eine starke Ähnlichkeit zwischen dem Namen und der Dokumentation gibt und im zweiten Fall überhaupt keine Gemeinsamkeiten existiert. Die Implementation der Metrik bewertet in beiden Fällen die Dokumentation mit 0 Punkten, in allen anderen Fällen vergibt sie 100 Punkte.

	\begin{figure}[ht!]
			\lstinputlisting
			[caption={Zwei Methoden mit mangelhafter Kohärenz},
			label={lst:coherence_example},
			captionpos=b,language=java, basicstyle=\footnotesize, tabsize=1, showstringspaces=false,  numbers=left]
			{figures/metrics/coherence.java}
	\end{figure}
		
 Listing \ref{lst:coherence_example} zeigt zwei Methoden, bei denen die Kohärenz nicht gut ist. Bei der ersten Methode ist der Kohärenzkoeffizient 0,75, da alle Wörter der allgemeinen Beschreibung der Komponente (Z. 2) außer \enquote{the} im Komponentennamen auftreten. Zur Verbesserung der Kohärenz könnte beispielsweise erwähnt werden, wie sich die Methode bei negativen Werten verhält. Bei der zweiten Methode gibt es überhaupt keine Gemeinsamkeiten zwischen den Namen der Komponente und der Dokumentation (Z. 11), sodass der Kohärenzkoeffizient gleich 0 ist. Hier könnte z.~B. zur Verbesserung  das Wort \enquote{absolute} in der Dokumentation erwähnt werden.

Die Kohärenzmetrik hilft dem Entwickler dabei, dass die Dokumentation zu dem Namen der dokumentierten Komponente passt und gleichzeitig einen Mehrwert bietet, sodass die Dokumentation nützlich ist. Allerdings können Synonyme den Wert verfälschen, da dann keine gemeinsamen Wörter gefunden werden. Auch durch Füllwörter (wie z.~B. \textit{the}, \textit{of} etc.) kann die Bewertung unterschätzt werden.

\subsection{Metriken, welche  nach Fehlern suchen}\label{chapter:metrics_errors}

Die letzte Kategorie an Metriken analysiert die Dokumentation und sucht nach dem Vorkommen von bestimmten Fehlern, welche die Qualität der Dokumentation negativ beeinflussen können. Solche Fehler können beispielsweise Rechtschreibfehler oder fehlerhafte Formatierung (wie z.~B. nicht konformes \ac{HTML}) sein. In der offiziellen Javadoc-Dokumentation wird außerdem geraten, bestimmte Abkürzungen (wie z.~B. \textit{e.~g.}, \textit{aka.} oder \textit{i.~e.}) zu vermeiden \cite{HowtoWriteDocCommentsfortheJavadocTool}. 

		\begin{figure}[ht!]
			\lstinputlisting
			[caption={Methode mit unvollständiger Erklärung der Behandlung von Nullwerten},
			label={lst:null_handling_example},
			captionpos=b,language=java, basicstyle=\footnotesize, tabsize=1, showstringspaces=false,  numbers=left]
			{figures/metrics/null_handling.java}
		\end{figure}
		
Bei Methoden sollte zudem darauf eingegangen werden, ob die Methode \textit{null} als Parameterwert akzeptiert oder \textit{null} zurückgeben kann, damit der Benutzer die Methode korrekt verwenden kann \cite{javadoc_coding_standards}. Listing \ref{lst:null_handling_example} zeigt eine Methode, bei denen die Behandlung von \textit{null} nur teilweise erwähnt wird. Beim ersten Parameter (Z. 3) wird erwähnt, dass der Parameter niemals null sein kann und daher ein Nullwert zu einer Ausnahme führen kann.  Beim zweiten Parameter (Z. 4) ist dies nicht klar und der Benutzer der Dokumentation wird im Unklaren gelassen, was zu einer fehlerhaften Verwendung der Methode führen kann.

In allen Fällen sollte ein Verstoß gegen diese Konventionen zu Punkteabzug führen.
Dazu können pro Verstoß Fehlerpunkte vergeben werden. Diese Fehlerpunkte können dann der Funktion 
\begin{equation}
     B(l)=S-(S-B_0)*e^{-k*l}
 \end{equation} 
 
 übergeben werden, die eine beschränkt fallende Exponentialfunktion ist. Dabei ist $S$ die untere Schranke für die Bewertung, $B_0$ die bestmögliche Bewertung,  $k$ eine Konstante für die Wachstumsrate und $l$ die Zahl der Fehlerpunkte. Durch diese Funktion erhalten Kommentare mit vielen Fehlerpunkten eine geringere Bewertung als Kommentare mit keinen oder wenigen Fehlerpunkten. Die Wachstumsrate $k$ kann dabei parametrisiert sein und sollte auch von der jeweiligen Fehlerkategorie abhängen. So kann beispielsweise argumentiert werden, dass Rechtschreibfehler weniger gravierend sind als fehlerhaftes \ac{HTML}. 
 
 Bei allen Metriken sollte allerdings beachtet werden, dass diese nur nach relativ einfachen Fehlern suchen. Bei einer Prüfung des \ac{HTML}-Inhaltes werden beispielsweise nur nicht geschlossene Tags detektiert. Auch falsch-positive Ergebnisse können auftreten, wenn beispielsweise bei der Rechtschreibprüfung ein nicht im Wörterbuch vorkommendes Wort, das aber korrekt geschrieben ist, bemängelt wird.   
 
 Als weitere Möglichkeit, die in der Literatur nicht erwähnt wird, könnten lange Methoden, die bestimmte Fehler aufweisen, härter bestraft werden. So kann beispielsweise argumentiert werden, dass lange Methoden, die sowieso vermieden werden sollten \cite[S.~34]{martin2009clean}, eher dokumentiert werden sollten, da sie komplexer sind. Daher können längere undokumentierte Methoden schlechter bewertet werden, indem sie mit Fehlerpunkten, welche äquivalent zu ihren \ac{LOC} sind, bestraft werden. 

 \section{Algorithmen zur Bildung eines Gesamtergebnisses}\label{chapter:algos_aggregation}
Zur Bildung eines Gesamtergebnisses aus verschiedenen Teilresultaten gibt es verschiedene Möglichkeiten, die in diesen Abschnitt vorgestellt werden. Jeder vorgestellte Algorithmus hat bestimmte Voraussetzungen und  kann eine große Auswirkung auf die Interpretation des Gesamtergebnisses haben, sodass die Wahl sorgsam überlegt werden muss. 

\subsection{Klassische Algorithmen}
Die klassischen Algorithmen zur Berechnung eines Ergebnisses aus mehreren Ergebnissen sind hinlänglich bekannt. Sie alle basieren auf der Annahme, dass die dahinter liegende Verteilung symmetrisch ist. Im Kontext der besprochenen Metriken bedeutet dies, dass die Wahrscheinlichkeit für eine gute Bewertung genauso sein sollte wie eine schlechte Bewertung. Bei vielen Metriken zur Bewertung der Softwarequalität kann diese Annahme nicht getroffen werden \cite[S.~313]{Youcantcontroltheunfamiliar:Astudyontherelationsbetweenaggregationtechniquesforsoftwaremetrics}. Auch bei den vorgestellten Metriken muss diese Annahme nicht zwangsläufig stimmen, sodass die Gesamtbewertung stets kritisch hinterfragt werden sollte. 
\subsubsection{Arithmetischer Mittelwert}
Der arithmetische Mittelwert wird durch die Klasse \textit{MetricResultBuilder} implementiert. Dieser Algorithmus berücksichtigt jedes Ergebnis gleichermaßen. Dies ist insbesondere dann sinnvoll, wenn keine guten Kriterien für die Gewichtung von Metriken, Dateien oder Komponenten gefunden werden können. Es sollte allerdings auch beachtet werden, dass der Mittelwert extreme Ausreißer berücksichtigt. Dies kann hier sinnvoll sein, da eine in Teilen sehr schlechte Dokumentation besser berücksichtigt wird und so ein verlässliches Gesamtergebnis geliefert wird.


\subsubsection{Median}
Der Median der Einzelresultate wird von der Klasse \textit{MedianResultBuilder} berechnet. Dabei werden die Einzelresultate nach dem bekannten Median-Algorithmus verarbeitet. Bei einer geraden Anzahl an Elementen wird der Median aus dem Mittelwert der zwei infrage kommenden Ergebnisse gebildet. 

Der Median berücksichtigt einzelne Ausreißer nicht und kann daher interessant sein, wenn ein allgemeines Bild von der Dokumentationsqualität erhalten werden soll. Es sollte aber beachtet werden, dass die Anwendung des Medians ein Sortiervorgang benötigt, der in den meisten Fällen eine Komplexität von $O(n*log(n))$ hat.  Zudem kann sich der Median stark ändern, wenn einzelne Komponenten hinzugefügt oder entfernt werden.


\subsubsection{Gewichteter Mittelwert}\label{chapter:weighted_aggreg}
Der gewichtete Mittelwert ist in der Klasse \textit{WeightedMetricResultBuilder} implementiert. Die Zuweisung der Gewichte erfolgt, wie in Kapitel \ref{chapter_weights_assign} beschrieben, durch einen \textit{WeightResolver}. Die Gewichte müssen nicht normiert werden, da dies während der Berechnung implizit erledigt wird. Die Resultate jeder Metrik werden multipliziert mit dessen Gewicht, dann aufsummiert und zuletzt durch die Summe aller Gewichte geteilt. 


Dieser Algorithmus ermöglicht es zum Beispiel, bestimmte Metriken zu bevorzugen bzw. zu benachteiligen. Dies ist sinnvoll, da nicht jede Metrik immer ein aussagekräftiges Ergebnis liefert und bestimmte Metriken je nach Situation ein besseres Bild über die Dokumentationsqualität liefern. Allerdings ist auch zu beachten, dass die Wahl der Gewichte nicht trivial ist und ein Vergleich von Ergebnissen, die verschiedene Gewichte verwenden, nicht sinnvoll ist.

\subsubsection{Gewichteter Median}
Der gewichtete Median wurde leicht abweichend nach \cite[S.~37]{YAGER199835} implementiert. Dabei wird zunächst die Summe der Gewichte berechnet und die Resultate nach ihrem Gewicht sortiert. Anschließend werden die sortierten Resultate und ihre Gewichte so lange aufsummiert, bis diese temporäre Summe die Hälfte der Gesamtsumme überschreitet. Das Metrikergebnis, bei der diese Bedingung zutrifft, ist das gesuchte Gesamtergebnis. Die Vor- und Nachteile dieses Algorithmus entsprechen den Vor- und Nachteilen des gewichteten Mittelwerts und des Medians. So muss auch hier eine Sortierung durchgeführt werden und die Wahl der richtigen Gewichte ist nicht trivial. 

 \subsection{Algorithmen aus der Ökonomie}
 
 Als Alternative zu den klassischen Algorithmen wurden Verfahren aus den Wirtschaftswissenschaften vorgestellt, die eine bessere Aggregierung der Einzelergebnisse ermöglichen sollen. Diese Verfahren werden in der Ökonomie benutzt, um die Ungleichheit von Einkommen mathematisch darzustellen. So besitzen in den meisten Staaten viele Menschen nur wenig Einkommen und nur einige Menschen ein sehr hohes Einkommen. Auch auf Softwareprojekten lässt sich diese Erfahrung übertragen, da es beispielsweise nur wenige Methoden mit sehr vielen Codezeilen geben wird, aber dafür sehr viele Methoden mit einer relativ kleinen Anzahl an Zeilen. Beispiele für diese Algorithmen sind Gini, Theil oder Atkinson. Einige implementierte Metriken  basierend auf die Anzahl der \ac{LOC}, sodass eine Anwendung der klassischen Algorithmen problematisch sein kann, wenn die Verteilung sehr asymmetrisch ist. \cite[S.~314]{Youcantcontroltheunfamiliar:Astudyontherelationsbetweenaggregationtechniquesforsoftwaremetrics} 
 
 Allerdings beschreiben diese Algorithmen nur, ob die Ergebnisse der Metriken ungleich verteilt sind und nicht, wie gut die Ergebnisse sind. Ein Projekt, bei dem alle Metriken schlechte Resultate liefern, würde dann ein ähnliches Gesamtergebnis liefern als ein Projekt, welches überall gut bewertet wird \cite[S.~1121]{Softwarequalitymetricsaggregationinindustry}.  Im Kontext der Bachelorarbeit könnte dies bedeuten, dass eine nicht vorhandene Dokumentation genauso gut bewertet wird wie eine exzessive Kommentierung aller Komponenten. 
 
 Daher werden diese Algorithmen vom Tool in der ausgelieferten Fassung nicht implementiert. Sie können aber in Verbindung mit anderen Aggregationsalgorithmen interessant sein, um mehr über eventuelle Ungleichheiten bei der Dokumentation zu erfahren.
 
 \subsection{Squale}\label{chapter:squale}
 Eine andere Möglichkeit zur Aggregierung von Metrikergebnissen wird in \cite[S.~1124ff.]{Softwarequalitymetricsaggregationinindustry} vorgestellt. Das sogenannte Squale-Modell berechnet aus verschiedenen Teilergebnissen von Metriken ein Gesamtergebnis und kann dabei besonders schlechte Ergebnisse stärker gewichten, damit eventuelle Probleme schnell erkannt werden können. 
 
 Im ersten Schritt werden die Ergebnisse der einzelnen Metriken, die auf unterschiedliche Skalen basieren, auf einer Skala normalisiert. Die Autoren nutzen dafür das Intervall 0 bis 3, da so eine Einteilung in Ziel nicht erfüllt (0 bis 1), Ziel größtenteils erfüllt (1 bis 2) und Ziel erfüllt (2 bis 3) möglich ist. Diese normalisierten Ergebnisse werden von den Autoren als \enquote{Individual marks (IM)} (z.~dt. individuelle Bewertungen) bezeichnet \cite[S.~142]{AnEmpiricalModelforContinuousandWeightedMetricAggregation}. 
 
 Im nächsten Schritt werden die einzelnen Ergebnisse aggregiert. Dazu wird jeder IM der Funktion
 \begin{equation}
     g(\text{IM})=\lambda^{-\text{IM}}
 \end{equation} übergeben. Dabei ist $\lambda$ eine Konstante, welche bei der Gewichtung der Ergebnisse hilft. Sollen schlechte Ergebnisse sehr hart bestraft werden, sodass bereits einige Fehler zu einer schlechten Bewertung führen, so muss $\lambda$ eine große Zahl sein. Durch die Wahl einer kleineren Zahl kann die Bewertung toleranter ausfallen. Aus den Ergebnissen der Funktion $g(IM)$ wird anschließend der Mittelwert $W_{avg}$ gebildet.
 
 Im Anschluss daran muss der Mittelwert wieder in den ursprünglichen Wertebereich abgebildet werden. Dazu wird die Umkehrfunktion
 \begin{equation}
     g^{-1}(W_{avg})=-\log_\lambda (W_{avg})
 \end{equation} angewendet. Dieses Ergebnis wird von den Autoren als \enquote{global mark (GM)} (z.~dt. globale Bewertung) bezeichnet und repräsentiert das endgültige Ergebnis.
 
 Durch diese Aggregierung werden Komponenten mit schlechten Ergebnissen stärker gewichtet und haben so einen größeren Einfluss auf das Gesamtergebnis. Allerdings kann dies bei der Dokumentation dazu führen, dass nur wenige schlecht dokumentierte Komponenten die Bewertung stark nach unten ziehen, sodass das Erreichen einer guten Bewertung einen hohen Aufwand erfordert. 
 
 Im Tool wird dieser Algorithmus ebenfalls unterstützt. Die Konstante $\lambda$ kann über den Konfigurationsparameter \textit{builder\_params} festgelegt werden. Standardmäßig ist er 9, was für eine mittlere Bestrafung von schlechter Dokumentationsqualität führt \cite[S.~1127]{Softwarequalitymetricsaggregationinindustry}. Intern werden die Teilergebnisse in das Intervall 0 bis 3 konvertiert, da ein Intervall von 0 bis 100 zu einer hohen negativen Potenz führen kann und dementsprechend mit Rundungsungenauigkeiten behaftet ist. Am Ende wird das Ergebnis wieder in das originale Intervall von 0 bis 100 gebracht. 
 
 
 

 \end{comment}
 