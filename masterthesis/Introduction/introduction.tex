\label{sec:introduction}
Code smells are a major issue  in modern software development as they tend to induce bugs and increase costs in the future \cite{mealyEvaluatingSoftwareRefactoring2006}.

One example of code smells is data clumps \cite{BaumgartnerAP23}. These can be defined as a group of variables that are used in multiple parts of a software project \cite{fowler2019refactoring}. For instance, if the variables \textit{x}, \textit{y}, and \textit{z} are used multiple times in the source code, they could be interpreted as an Euclidean point or vector. Therefore, one could extract  a class \textit{Vector3d} containing the fields x, y, and z. 

This refactoring approach reduces the code side, makes the code more readable, and simplifies further changes to the source code \cite{data_clumps_refactoring_guru} \cite{join_data_items}. 

There are many approaches to detecting code smells (e.g. SonarCloud,  PMD, Checkstyle) that can be integrated into the development process but these do not automatically fix them  \cite{vidalApproachPrioritizeCode2016}. 
Since developers are often distracted with implementing new features, fixing bugs, or doing similar tasks, the refactoring of code smells gets pushed back so that many code smells (even if detected) remain unfixed.  \cite{10.1145/2393596.2393655}.


One approach to solve this issue is to automatically fix certain code smells that are easy to define so that human intervention is minimized. This automation can be regularly applied, allowing code smells to be gradually addressed without distracting developers from their primary tasks but profiting from cleaner code. 
However,  automatic refactoring  is more difficult than a simple detection because it requires changes to the source code that have the potential to induce bugs or even make the software project unable to build \cite{9796303}. Hence, the tools used must be carefully assessed. 

The data clump example mentioned above is one of those code smells that can be clearly defined. For instance, one can propose that a group of more than three variables that also appear in more than three different parts of the source code constitutes a data clump \cite{zhangImprovingPrecisionFowler2008}.




However, a successful re-factorization of data clumps requires several steps. This includes renaming of identifiers, removing symbols, and extracting a class. Also a class name for the extracted class (e.g. \textit{Vector3}) must be determined. To minimize human intervention, a suitable identifier of the class must be found that accurately describes the purpose of these variables and their connection. Hence, domain knowledge and some creativity are necessary to fully perform the refactoring process.

As a result, additional tools (such as ChatGPT \cite{ChatGPT_url}) are needed to fully automate the refactoring pipeline while minimizing the need for manual changes. 


\section{Objective}


The goal of this master thesis is to develop a tool that  combines ChatGPT and other refactoring tools  to automatically detect and refactor data clumps in software projects. The program  shall at least support the Java programming language but shall be extendable to  other programming languages. The tool shall also  be able to filter out some files and data clumps by several criteria to reduce resources and costs. 

The methodology of this master thesis will be evaluated by sending  pull requests about discovered data clumps and a refactoring proposal to several public GitHub repositories and analyzing whether the pull request is accepted, rejected, or amended.   

\section{Approach}

A major part of the master thesis will deal with how to use and integrate ChatGPT into the refactoring pipeline. 
ChatGPT is an AI language model developed by OpenAI that uses a Generative Pre-trained Transformer (GPT) model to process textual input data (i.e. natural language or source code). Users can provide queries, questions, or other textual material to ChatGPT and the model responds with a textual reply attempting to satisfy the user's request \cite{yetistirenEvaluatingCodeQuality2023}. It also employs a conversation feature so that previous requests and replies can be referred to by future requests and responses \cite{sobania2023analysis}.

Since ChatGPT can also process source code \cite{sadik2023analysis}\cite{guo2023exploring}, one goal of this master thesis is to test to what extent it can help developers find data clumps and refactor them. Different extents of ChatGPT inclusion will be tested  

With a minimal ChatGPT inclusion approach,  data clumps will be found using traditional approaches and ChatGPT will be provided with a list of data clump variables and asked to suggest a suitable name for the extracted class, while the refactoring process will be executed using other refactoring tools.


Conversely, it will be tested whether ChatGPT can execute the refactoring itself by providing the whole source code of a software project and providing specific queries to find the data clumps, refactor them, and output the refactored source code. \cite{White2023ChatGPTPP}.

Each task of the data clump detection and refactoring pipeline will be assigned to one tool, and the intermediate results of one tool will be used later for the succeeding tasks in the pipeline.

The goal is to find a ChatGPT usage that finds most data clumps, while ensuring that they are refactored correctly. Also possible costs for the usage of ChatGPT and other resources will be considered \cite{xia2023conversation}. \cite{4ef0b456377aafb68884e643779dffb36b8e7cc1}.

\begin{comment}
Ein wichtiger Bestandteil der Softwareentwicklung von heute ist die Softwaredokumentation. Dies liegt unter anderem daran, dass die Größe von Softwareprojekten steigt, sodass die Entwickler schnell den Überblick über das Projekt verlieren können und daher zusätzliche Informationen neben dem Code benötigen \cite[S.~1]{StaticAnalysis:AnIntroduction:TheFundamentalChallengeofSoftwareEngineeringisOneofComplexity.}. Nichtsdestotrotz wird die Softwaredokumentation von Entwicklern oft vernachlässigt \cite[S.~83]{Qualityanalysisofsourcecodecomments}.  Die Gründe für schlechte Dokumentation sind vielfältig. Das Schreiben der Dokumentation wird oft als mühevoll empfunden und erfordert Fähigkeiten, die ein Programmierer nicht zwangsläufig besitzt \cite[S.~70]{AutomaticQualityAssessmentofSourceCodeComments:TheJavadocMiner} \cite[S.~593]{Softwareengineeringandsoftwaredocumentation:aunifiedlongcourse}.  

Weitere Studien verdeutlichen die Problematik der mangelhaften Softwaredokumentation. So belegt eine Umfrage aus dem Jahr 2002 mit 48 Teilnehmern  beispielsweise, dass die Dokumentation  bei Änderungen am System  nur mit Verzögerung angepasst wird. Knapp 70~\% der Teilnehmer stimmen der Aussage zu, dass die Dokumentation immer veraltet ist.   \cite[S.~28--29]{TheRelevanceofSoftwareDocumentationToolsandTechnologies:ASurvey}

Eine weitere Studie  \cite[S.~1199--1208]{SoftwareDocumentationIssuesUnveiled} aus dem Jahr 2019 verdeutlicht viele Aspekte aus der vorgenannten Umfrage. Es wurden dabei Daten aus Stack Overflow, GitHub-Issues, Pull-Requests und Mailing-Listen automatisiert heruntergeladen und dann von den Autoren analysiert, ob und inwieweit diese durch mangelhafte Softwaredokumentation verursacht wurden.  Die Studie belegt, dass von 824 Problemen, die etwas mit dem Thema \enquote{Softwaredokumentation} zu tun haben, 485 sich auf den Inhalt der Dokumentation beziehen (wie z.~B. unvollständige, veraltete oder sogar inkorrekte Dokumentation). Bei 255 Einträgen gab es Probleme mit der Struktur der Dokumentation, sodass beispielsweise Informationen schlecht auffindbar sind oder nicht gut verständlich sind.


Eine andere Umfrage aus dem Jahr 2014 mit 88 Teilnehmern zeigt, dass eine automatisierte Überprüfung der Dokumentationsqualität von knapp der Hälfte der befragten Entwickler gewünscht wird. Die Autoren der Studie sehen dies als Zeichen dafür, dass ein grundsätzliches Bedürfnis zur automatisierten Bewertung von Dokumentationen besteht und daher weitere Studien notwendig sind. \cite[S.~340]{TheValueofSoftwareDocumentationQuality}

Die mangelhafte Dokumentation führt dazu, dass nicht nur nachfolgende Entwickler Probleme mit dem Codeverständnis haben, sondern auch Entwickler eines Moduls nach einer längeren Pause Zeit aufbringen müssen, um den Code wieder zu verstehen \cite[S.~511]{vestdam}.  Auch für Kunden/Auftraggeber ist eine gute Dokumentation wichtig, da gut dokumentierte Software tendenziell besser wartbar ist und somit mehr Nutzen bringt \cite[S.~83]{Qualityanalysisofsourcecodecomments} \cite[S.~1]{SoftwareDocumentationManagementIssuesandPractices:ASurvey}.



\section{Zielsetzung}
Aufgrund der Relevanz von gut dokumentierter Software ist eine regelmäßige Rückmeldung über die Dokumentation von hoher Bedeutung. Spezielle Metriken, die eine numerische Auskunft über die Qualität der Softwaredokumentation liefern, sind eine Möglichkeit, diese Rückmeldung zu geben. Diese Metriken verschaffen dem Programmierer eine Einschätzung darüber, ob die Softwaredokumentation ausreichend ist oder eine Verbesserung sinnvoll wäre. Die Qualität der Softwaredokumentation kann auf unterschiedliche Art und Weise bewertet werden. So kann beispielsweise die bloße Existenz einer Dokumentation geprüft werden oder aber auch die Verständlichkeit der Dokumentation bewertet werden, daher kann es sinnvoll sein, mehrere Metriken zu verwenden \cite[S.~29]{pfleeger1992using}. Damit ein Entwickler einen Gesamtüberblick über die Dokumentationsqualität erhält, können diese Metriken kombiniert werden, um eine einzelne numerische Bewertung der Qualität der Dokumentation zu erhalten. 
Dabei ist es auch ratsam, die Metriken zu gewichten oder eine andere Methode zur Kombination der Metrikergebnisse zu benutzen, weil nicht jede Metrik die gleiche Zuverlässigkeit und Relevanz besitzt \cite[S.~1117ff.]{Softwarequalitymetricsaggregationinindustry}.

Damit das Feedback über die Softwaredokumentation auch wahrgenommen wird, sollte die Qualität regelmäßig  überprüft werden. Dies kann automatisiert im \ac{CI/CD}-Prozess erfolgen, bei dem Software kontinuierlich getestet und für den Release (z.~dt. Veröffentlichung) vorbereitet werden kann. Durch CI/CD können Unternehmen effizienter und besser Software entwickeln. So konnte das Unternehmen \textit{ING NL} die gelieferten Function-Points vervierfachen und die Kosten für einen Function-Point auf einen Drittel reduzieren \cite[S.~520]{Vassallo2016}.

\hfill

Basierend auf diesen Überlegungen soll ein Tool (z.~dt. Werkzeug) entwickelt werden. Dieses Tool (im Folgenden auch \textit{DocEvaluator} soll ein gegebenes Software-Projekt analysieren und eine numerische Bewertung abgeben, die eine heuristische Aussage über die Qualität der Softwaredokumentation trifft.  Dabei soll das Tool primär für Javadoc und Java bis Version 8 konzipiert werden, allerdings soll während der Entwicklung auch darauf geachtet werden, dass eine Portierung auf eine andere Programmiersprache ermöglicht wird und die Bewertung der Dokumentation unabhängig von der Programmiersprache funktioniert. Außerdem wird zur Vereinfachung nur englischsprachige Dokumentationen betrachtet. Komplexe \ac{NLP}-Metriken sollen dabei außer Acht gelassen werden. Auch Verfahren, die  den  Quellcode mit der Dokumentation vergleichen, wie z.~B. \textit{iComment} in \cite[S.~145ff.]{icomment}, sollen unberücksichtigt bleiben, da sie im Rahmen dieser Bachelorarbeit zu komplex sind. 

Dabei sollte es nicht unbedingt das Ziel sein, dass jede Komponente dokumentiert ist, sondern dass die wichtigen Komponenten eine gute Dokumentationsqualität haben und somit die Wartung vereinfacht wird. Als Komponente im Sinne dieser Bachelorarbeit werden dabei Klassen, Schnittstellen, Methoden und Felder verstanden. 

Dieses Tool soll anschließend in den \ac{CI/CD}-Prozess eingebunden werden, sodass die Dokumentationsqualität kontinuierlich geprüft werden kann. Als \ac{CI/CD}-Plattform soll dabei \textit{GitHub Actions} \cite{GithubActions} verwendet werden, da GitHub von der Mehrzahl der Entwickler und großen Unternehmen verwendet wird \cite{github_popular}. Mittels GitHub Actions soll das Tool bei einer sehr schlechten Dokumentationsqualität den Entwickler auf diesen Umstand hinweisen, indem beispielsweise ein Merge (z.~dt. Verschmelzung) in GitHub verhindert wird. Auch bei einer deutlichen inkrementellen Verschlechterung der Qualität soll der Entwickler informiert werden, um so eine ausreichende Qualität der Dokumentation sicherzustellen. 

Ein Forschungsziel dieser Bachelorarbeit ist es zu prüfen, wie das Programm konzipiert werden muss, um mehrere Programmiersprachen zu unterstützen. Ein weiteres Ziel der Arbeit beschäftigt sich mit der Frage, wie die Ergebnisse der Metriken kombiniert werden können, um eine präzise Aussage über die Gesamtqualität der Dokumentation eines Softwareprojektes zu erhalten. Die Konzeption einer Architektur, mit der weitere Metriken hinzugefügt werden können und der Nutzer des Tools auswählen kann, welche Metriken bei der Bewertung der Dokumentationsqualität berücksichtigt werden sollen, soll ebenfalls als Forschungsziel untersucht werden. Zuletzt soll als Forschungsfrage diskutiert werden, welche Metriken eine heuristische Aussage über die Qualität der Dokumentation treffen können. 


\section{Gliederung}
In Kapitel \ref{sec:background} werden die wichtigen Grundlagen über die Themen dieser Bachelorarbeit erläutert. Dazu  wird zunächst der Begriff Softwaredokumentation definiert und ein Bezug zu Code-Smells hergestellt. Mittels Javadoc wird dann erläutert, wie Software dokumentiert werden kann. Anschließend wird eine Einführung in GitHub Actions gegeben.  Zudem wird eine Einführung in ANTLR4 gegeben, das für das Parsing der Quellcodedateien in Java verwendet wird. Zuletzt werden einige wissenschaftliche Arbeiten mit vergleichbaren Zielsetzungen präsentiert und Tools vorgestellt, die ebenfalls die Qualität der Softwaredokumentation bewerten können.

In Kapitel \ref{chapter_conception} werden die Fragestellungen besprochen, die sich beim Design des Tools ergeben haben. Dazu gehören die notwendigen Objekte und ihre Interaktion untereinander und wie von einer losen Ansammlung von Dateien zu einer Bewertung der Softwaredokumentation gelangt werden kann.

In Kapitel \ref{chapter:program} wird anschließend erläutert, wie aus dieser Konzeption ein vollständiges Programm entwickelt wird. Dazu wird erläutert, wie das Programm in GitHub Action eingebunden werden kann. Im Anschluss daran wird ein Überblick über die implementierten Metriken mit ihren Vor- und Nachteilen gegeben. Außerdem werden die Algorithmen bzw. Verfahren erläutert, um die Ergebnisse der einzelnen Metriken zu einem Gesamtergebnis zu aggregieren. 

In Kapitel \ref{sec:evaluation} wird das Programm dann mit ähnlichen Tools verglichen, indem beispielhafte Java-Projekte aus GitHub mit allen Programmen analysiert werden und die Geschwindigkeit und die Qualität jedes Programmes ermittelt wird. 

Im abschließenden Kapitel wird der Inhalt der Arbeit zusammengefasst und ein Fazit gezogen. Es werden offengebliebene Fragen beleuchtet und ein Ausblick gegeben, welche Möglichkeiten zur Verbesserung des Tools sinnvoll wären. 
\end{comment}