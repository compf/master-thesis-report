
\begingroup
\renewcommand{\cleardoublepage}{} % TODO maybe removing this and next
\renewcommand{\clearpage}{}
\chapter{Implementation}\label{chapter:implementation}
\endgroup
In this chapter, the theoretical thoughts of chapter \ref{chapter_conception} will be implemented. Firstly, in section \ref{sec:config}, the configuration of the tool is explained. After that experiments are performed in order to find out how ChatGPT can be integrated for data clump finding ( see section \ref{sec:prompt_engineering_impl}). In section \ref{sec:step_impl} an overview will be given which combination of tools can be used together to find and refactor data clumps and the limitation and strengths of each combination will be outlined. 
\section{Configuration}\label{sec:config}
An important aspect for the usability of the tool is the possibility to configure the tool for the user's need. 

Since the goal of the tool is to allow the combination of multiple services and other tools in order to find and refactor data clumps, the user must be able to define which handler deals with which step (see section \ref{sec:pipeline}).

The configuration is provided by a \ac{JSON} file whose location needs to be provided to the tool via a command line argument. It might be argued that providing the configuration directly via the command line is better suited than a separate configuration file because they do not require the creations of files and are easier for users who start the tool just once. Nevertheless, configuration files are persistent and escpecially \ac{JSON} can be more easily structured so that they are easier to understand. As a result, only \ac{JSON} files will be used  for the configuration. 


Listing \ref{lst:config} shows an example configuration file:
  \begin{figure} [htbp!]
			\lstinputlisting
			[caption={ Example configuration file},
			label={lst:config},
			captionpos=b, basicstyle=\footnotesize, tabsize=2, showstringspaces=false,  numbers=left]
			{figures/chapter4/config.json}
		\end{figure}


In the beginning, the programming language is defined (l.~1). Then, the user can specify which handler should handle which step (l.~3-26). In this example, there are two steps. 

The first step (l.~4-9) establishes the project location and therefore the code obtaining context.

The second step (l.~ 10-26), deals with the detection of data clumps. Here a handler is used that allows detection and refactoring of data clumps via ChatGPT. This specific handler requires sub-handlers that perform intermediate task. For instance, the \textit{SimpleInstructionHandler} loads an instruction from a file path. The \textit{PairOfFileContent} handler submits pairs of file contents to ChatGPT (see section \ref{sec:initial_experiments}).

It should be noted that the order of the handlers in the configuration  does not matter because the execution order is constant and in most cases, each step requires the context of a previous step so that parallel execution or vice-versa execution of steps is not possible. Only in the case of usage finding and name finding would a parallel execution make sense because no step is dependent on the other.  However, this exception is not implemented.

Not all relevant objects are part of the pipeline. Some are outside of the pipeline and can be referenced by all handlers. Here, the large language model \ac{API} is initialized with ChatGPT(l.~ 28).

Each object that appears in the \ac{JSON} is instantiated using dependency injection. This means that the objects initially are registered at a central location but not fully instantiated until they are really needed. The general approach is as follows:
\begin{enumerate}
    \item The main program loads the \ac{JSON} and iterates over all objects
    \item Each object has a category (e.~g. \textit{LanguageModelInterface} (see l.~ 28)) and a name (e.~g. ChatGPTInterface) 
    \item The name of the object and it paremters are registered at the central dependency manager under the category given
    \item If a handler or other object needs another object of a specific category, it can ask the central manager. If the respective object has  yet not been created, it will be created. Otherwise, the already instantiated object will be returned. 
\end{enumerate}
With this approach, a looser coupling can be achieved. The main program and the handlers does not need to know the exact details of the configuration but provide and receive them from a central dependency manager. Only the handler-specific configuration (e.~g. loading the project location  for the \textit{CodeObtaining} handler is still required and must be performed by each handler. These parameters are stored by the dependency manager. 


\section{Prompt engineering ChatGPT for data clump detection}\label{sec:prompt_engineering}

As outlined in  section \ref{sec:llm_challenges}, using \ac{LLM} can result in some challenges for refactoring code. Therefore, experiments are needed to test which input can lead to good results for finding or refactoring data clumps or both. These experiments should not be seen as a replacement for the full evaluation discussed in chapter \ref{chapter:eval}, but a mandatory prerequisite to save costs since the full evaluations needs to be performed on a larger set of projects.

The input project is a modified version of the source code discussed in \ref{sec:data_clump_def} which is fairly small but include at least three data clumps. Table \ref{tbl:javaTest_data_clumps} lists all data clumps which exists in this project. The names in the \textit{From} or \textit{To} column are in the format \textit{className.methodName} or \textit{className} depending on whether it is a field-to-field \textit{(f2f} or parameter-to-parameter \textit{(p2p)} data clump

\begin{table}[]
    \centering
    \begin{tabular}{c|c|c|c|c}
         Id &From & To & Type & Data clump variables  \\\hline
         1 & MathStuff.printLength & MathStuff.printSum & p2p & x,y,z\\\hline
        2 & MathStuff.printSum & MathSum.printMax & p2p & x,y,z\\\hline
        3 & MathStuff.printLength & MathSum.printMax & p2p & x,y,z\\\hline
        4 & MathStuff & MathUser & f2f & sign, mantissa, exponent\\\hline
        5 & Library & MathUser & f2f & sign, mantissa, exponent\\\hline
        6 & MathStuff & Library & f2f & sign, mantissa, exponent\\\hline
    \end{tabular}
    \caption{All data clumps in the test projects}
    \label{tbl:javaTest_data_clumps}
\end{table}

In this section, the results of some of these experiments will be discussed.  As repeatedly noted, these experiments are not always reproducible and therefore are only to some extent helpful. In all cases, ChatGPT is asked to output in \ac{JSON} to have a consistent format.

In each experiment, different parameters of the input are adjusted. Each these parameters has an influence of the quality of the output:

\begin{itemize}
    \item Instruction
    \item Model
    \item Data format
    \item Data size
    \item Temperature
\end{itemize}

\subsubsection{Instruction}\label{sec:llm_instruction}
The instruction is a main part of a prompt to ChatGPT or other \ac{LLM}. It tells the  \ac{LLM} what to do and how to respond. As a result, varying the instruction is an important part of prompt engineering. \newline



Initially, it is useful whether ChatGPT can find data clumps without giving a specific definition. This means ChatGPT needs to deduce the definition of a data clump from its trained models which might not be the same as the definition used in section \ref{sec:data_clump_def}. For instance, data clumps are defined in this master thesis to have at least three common variables while other sources require at least four common variables. 
\begin{comment}
Nevertheless, the results are fairly good. ChatGPT ignores the inheritance between \textit{MathStuff} and \textit{BetterMathStuff} and does not output a data clump for each method in \textit{MathStuff}
\end{comment}


In another approach, ChatGPT is provided a specific definition of data clumps and asked to apply this definition on the project. Therefore, a precise definition is elementary for this approach.

The following query was used:
A data clump exists if
\begin{enumerate}
\item  Two methods (in the same or in different classes) have at least 3 common parameters
    and one of those methods does not override the other, 
\item At least three fields in a class are common with the parameters of a method (in the same or in a different class), or 
\item Two different classes have at least three common fields
\end{enumerate}


Another approach for including ChatGPT for data clump refactoring is example-based. Multiple examples of data clumps and how they can be refactored are presented to ChatGPT. In order to avoid any bias, each type of data clump is equally represented and field identifiers or parameters field identifiers are unique across the whole example. This results in a larger request size and thusly more costs. Nevertheless, if larger projects were to be used the size of the examples becomes negligible so that this approach has its reasons. The examples provided to ChatGPT can be found in the digital appendix under the path \enquote{data\_clump\_examples.java} 

\subsubsection{Model}
The model is a parameter for ChatGPT that controls the quality, performance, and cost of processing a query and responding to it. 

OpenAI provides multiple models for user of its API. Each  ChatGPT model  is based on the same original model but is fine-tuned to optimize for a specific task. For this master thesis, only  the  models \enquote{gpt-4-1106-preview} and \enquote{gpt-3.5-turbo-1106} are relevant. Both models support a \ac{JSON}-only mode, which means that any response by the model will be valid \ac{JSON} instead of natural language, so it easier to parse and process. \cite{chatgpt_new_models}

The model {gpt-4-1106-preview} is also called \textit{GPT-4 Turbo}. It can process 128,000 tokens at once and has a knowledge cutoff of April 2023. \cite{chatgpt_new_models}

The model \enquote{gpt-3.5-turbo-1106} has a context window size of 16,000. The knowledge cutoff is September 2021. \cite{chatgpt_new_models}


In general, GPT-4-Turbo has a better quality because of the larger context window size and more up-to-date information. While the more recent cutoff date might not be that important for data clump detection and refactoring, the larger context window size means that larger amount of code can be transmitted to ChatGPT at once. So, larger projects can be refactored more easily.  \cite{chatgpt_new_models}

On the other hand, the costs per 1000 tokens is  \$0.003 for GPT-3-Turbo whereas the costs for GPT-4 is \$0.01 for 1000 tokens. As a result, the GPT-4-Turbo is thrice as expensive as GPT-3 which can be a reason to use GPT-3. \cite{chatgpt_new_models}

Additionally, there are  stricter limits for using GPT-4. For instance, only 500,000 tokens can be sent per day, and 150,000 tokens can be sent per minute. In contrast, GPT-3-Turbo has  lower values for token per minutes but no  token per day limit.  ¸\cite{chatgpt_limits}

\subsubsection{Data format}

After creating a instruction, ChatGPT cannot do much because it does not have access to the source code which must be provided to ChatGPT. Therefore, the next part of the prompt is some representation of the source code

The two approaches tested are providing the complete source code and providing the \ac{AST}.

Providing the complete source code is the most natural manner to give ChatGPT the relevant information. Since ChatGPT is language-agnostic, it can deal with many programming languages and therefore is able to process the source code files to find data clumps. However, source code files can be large as they can contain comments, method bodies or other parts of the programming language that are not relevant for detecting data clumps. This leads to higher costs.

On the other hand, the \ac{AST} can be provided. The \ac{AST} only contains a reduced structure of the source code and will therefore cost not as much. However, the \ac{AST} is usually not available but must be produced by other tools or ChatGPT (which means the source code has to be sent to the \ac{LLM} nevertheless). Additionally, there is no fixed formate for the \ac{AST} which might complicate parsing and processing the \ac{AST} or the results by ChatGPT.

\subsubsection{Data size}

The size of the data is the amount of data that is submitted during a single request. Here also two approaches are tested.

In the first approach, the full project is submitted to ChatGPT to find data clump while in the second pairs of files are transmitted which each pair of file being a separate independent request.

The advantage of the first approach is that only one transaction needs to be done, and ChatGPT can find many data clumps in one shot. ChatGPT can therefore better understand the structure and provide better suggested name or find data clumps where the identifier names of the data clump variables are not identical. However,  a major issue is that many projects are too large, and therefore ChatGPT cannot provide the advantages just described.

In the second approach, pair of files are transmitted to ChatGPT with an instruction mentioned in \ref{sec:llm_instruction}
This means that every request only contains the instruction and the content of the two files. Therefore, ChatGPT can only find data clumps only in those two files and in no others as it has no memory of previous prompts. One might think that this approach is cheaper, however, if one still wants to find data clump in the whole software project, he has to to transmit $\binom{n}{2}=0.5*n*(n-1)$ where $n$ is the number of relevant files. Therefore, the cost can be much higher than transmitting each file only once as in the first approach.

Additionally, each file could be sent again separated in order to find data clumps that exists in a single file and might not be detected if two files are provided so that the total number of files would be $0.5*n*(n-1)+n$.

\subsubsection{Temperature}
As outlined in section \ref{sec:chatgpt}, the temperature is a significant parameter that controls the randomness of the \ac{LLM} output. A high temperature indicates that the response by ChatGPT is more random and unpredictable, while a low temperature results in more consistent responses.

Hence, a higher temperature can help ChatGPT to be more creative so that it may find data clumps that it would usually not detect. However, it can also lead to a lower specificity because it may flag wrong methods or fields as part of a data clump because of the increased unpredictability. 

As a result, adjusting the temperature parameter is a significant part for the small experiment. Here, two values will be tested. A low temperature of 0.1 and a hight value of 0.9. While these are two extremas and not a middle ground, they can give a good approximation of whether thelow or high temperatures are better for data clump detection. 

\begin{comment}
\subsection{Requesting follow-ups}

As outlined in section \ref{sec:chain of thought},  one prompt might no be enough to refactor data clumps. This might also apply for the simpler task of detecting data clumps.

Instead of expecting that a single request will detect all data clumps, another approach would be to repeat the same requests multiple times and aggregate the results. These follow-up prompts result in ChatGPT to re-analyze the code so that more data clumps can be found. 

The issue  with follow-up prompts is how often should a prompt be repeated. Each prompt repeat requires that the conversation history to be sent again because ChatGPT is stateless. This leads to higher resource usage and costs. There are multiple approach to implement follow up.

\begin{description}
\item[Full project follow-up] The whole project (all relevant files) is sent to  ChatGPT again with an instruction to find more data clumps. As a result, more data is sent per  request while the number of request can be reduced. As a  result, ChatGPt might find additional data clumps that it has not found during the previous prompts.
\item [Tuple-based follow-up] A tuple of two files in the project (i.~e. the content of those files) is sent again to ChatGPT with the instruction to find more data clumps. If the goal is to data clumps between tuples of all files, $0.5*n*(n-1)$ file tuples need to uploaded so that many requests will needed. However, this can help ChatGPT to focus on a smaller part of the software project to analyze in order to find more data clumps.
\end{description}
\end{comment}

\subsection{Results of the experiments}\label{sec:initial_experiments}

The results of the experiments are now described. Each parameter configuration tested three times. While more repetitions would be better for more valid results, this is nevertheless a small experiment whose goal is to find good and bad ChatGPT configurations. 

The following are the mein conclusion of the experiment:
\begin{itemize}
    \item GPT-4 leads to much better results than GPT-3
    \item GPT-4 and GPT-
    3 have better results if examples are provided instead of a data clump definition. However, the sensitivity of GPT-3 is better if definitions are used
    \item A high temperature is better for GPT-3. A high temperature for GPT-4 leads to reduced specificity but increased sensitivity. 
    \item While some results are very good(e-~g. 100 percent), the median and mean is often worse and seldom reach 50 percent. 
    \item The impact of providing all files or pair of files is negligible. 
    \item The daily limit of token for ChatGPT is a major issue during the experiments because it was frequently reached and therefore, the experiment  must be conducted over a longer period of time. 
\end{itemize}

As already noted, the resulkts should be interpreted cautiously because the project size is small. Nevertheless, some observations can be made:

\begin{itemize}
    \item Use Examples for data clump as they seem to produce better results
    \item Use a high temperature which allows for more creativity and apparent better results
    \item If possible (given the higher costs) use GPT-4
    \item Whether all files or pair of files should be transferred is difficult to say and the impact is difficult to determine. 
\end{itemize}
\label{sec:prompt_engineering_impl}
\section{Implementation of steps}\label{sec:step_impl}
While in section \ref{sec:pipeline} a general approach to find and refactor data clumps is given, the concrete implementation with the various tools is left out. In the following subsections, the implementations with the tools are described. In the case of \textit{code obtaining}, \textit{filtering}, only basic approaches are used so that they will not be detailed further. 


\subsection{Data clump doctor}

The  \textit{Data Clump Doctor} is NodeJS command line tool developed by Nils Baumgartner in preparation of this master thesis. It employs \textit{PMD} to find all classes, methods, and fields in a Java project to generate an \ac{AST}. This syntax tree is saved as a \ac{JSON}. 

In a second step, the generated \ac{AST} can be loaded again to find data clumps. This multi-step approach results in better performance since the detection of data clumps requires many nested for-loops so that any reduction in data size has a measurable effect on the performance.

All detected data clumps are reported in the format specified in section \ref{sec:data_clump_format}.

Since the general tool is developed in Typescript, the \textit{Data Clump Doctor} can easily be integrated.

\begin{comment}
\subsection{ChatGPT}
ChatGPT is another approach to detect data clumps as it can process code easily and report data clumps in any format the user wants. It also supports many programming languages that other tools do not provide.

However, ChatGPT has a limited context size, so processing large projects is  either too costly or simply not possible.

Giving ChatGPT the right instructions to find data clumps can also be challenging. While ChatGPT can define and find some data clumps without further context, it is better to give it a precise definition to work with. The following definition leads to good result, however it cannot be guaranteed that this will work forever:


A data clump exists
\begin{enumerate}
   
   \item if at least three fields also exists in another class
   \item if at least three fields also exists as method parameters in some method
    \item if two methods have at least three common parameters
\end{enumerate}
\end{comment}

\subsection{IntelliJ}
The Program Structure Interface provided by IntelliJ is an \ac{API} to analyze projects that IntelliJ can load. As a result, the various classes, methods etc. can be obtained which allows for the detection of data clumps. Like the ChatGPT approach, it can be easily extended to refactor the data clumps.

In order to use this API, an instance of IntelliJ must be started. IntelliJ can be started in a headless mode to reduce loading times and improve the performance so that no GUI is initialized. Nevertheless, IntelliJ requires many resources and much overhead, so that  the initialization needs some time.

Another problem during the development of the tool is the issue of correctly loading projects. While a project that has no IntelliJ meta data files can be loaded, finding references of a method or field can lead to undefined behavior. Sometimes all references are correctly found, sometimes only a subset of the references are found, and sometimes no even no references are detected. The reasons for this are difficult to determine and the documentation is scarce, so the the PSI approach seems to be only suitable for projects created via IntelliJ or correctly initialized by IntelliJ with the required meta data. Gradle and maven projects are therefore not suitable for the full refactoring step. 

\subsubsection{Data clump Refactoring}

IntelliJ proves to be a effective way to refactor data clump because of it easy access to the syntax tree.

There are two ways of using IntelliJ for the refactoring of data clumps.

In the first way, IntelliJ performs the refactoring with libraries that are already included. Noteworthy are the \textit{IntroduceParameterProcessor} and teh \textit{ExtractFieldsProcessor}. Both classes can perform all necessary steps to refactor data clumps. The former is used to refactor parameters in a method, while the latter is used for data clumps with fields. They can be combined in case of parameters-to-field data clumps. The problem is that -- as already noted-- IntelliJ only works correctly if the software project is an IntelliJ project which is not always teh case.

In case of a non-IntelliJ project like Gradle or Maven, a different approach is needed. Here, the \ac{LSP} as described in section \ref{sec:lsp} is used to find all references of the methods, fields, and variables that are part of the data clump. For instance, the \ac{LSP} find all references of a method parameter that is part of a data clump. These references are grouped into four categories. 
\begin{enumerate}
     \item A field is declared
    \item A variable (field or method parameter) is read from or assigned to
   
    \item A method is declared or overridden
    \item A method is called
\end{enumerate}

This information can be obtained by \ac{LSP} and is provided to IntellJ which does not have the capability to obtain the data in case of non-supported project.

IntelliJ can then perform the refactoring with the information it can obtain even from non-supported projects and the information from \ac{LSP}. For instance, the data from \ac{LSP} informas IntelliJ where a particular method call is located, and IntelliJ can parse the specific source code part so that refactoring is easier.

Now, IntelliJ can perform the refactoring. Deepending of the category of a reference determined in the previous step, IntelliJ needs to perform distinct step. 
\begin{enumerate}
    \item In case of a declared field, the field can be deleted because it is part of a field data clump. Now it can be determined whether the class contains already the new field replaces the fields of the data clump.
    \item If a method is declared, IntelliJ can modify the signature of the method. This can be the original method or an overridden one. IntelliJ needs to remove the parameters that are part of the data clump and add a new method parameter which replaces the method parameters of the data clump. 
    \item If a variable is used it can be replaced by a getter or setter call. For instance if the variable \textit{var} is read, and the name of the extracted class is \textit{Object}, any reading of the variable can be replaced by  \textit{object.getVar()} where \textit{object} is a variable of type \textit{Object} and \textit{getVar} is the getter of \textit{var}. Similarly, an assignment can be replaces by the setter.
    \item If a method is used, several substeps are needed.
    \begin{enumerate}
        \item First, for each argument provided in a method call it is determined whether the argument is connected to a data clump variable (i.~e. it provides a value to a parameter that is part of a data clump) 
        \item The position of those argument is stored and a reference to the argument is stored for further processing.
        \item Since the extracted class is known and existing, a matching constructor is determined to support all arguments to the data clump variables of the method call. 
        \item For each argument to a data clump variable, the argument is inserted into the matching position of the constructor, and the argument is deleted in the original method call. 
        \item the constructor call is added at the position of the method  where the parameter of the extracted class is expected. 
        
    \end{enumerate}
    
\end{enumerate}

This approach requires some coordination. For instance, the order of operation is important. Method and field declarations must be updated first because they are needed to perform the refactoring of variable usages and method calls successfully. 

Another aspect, where the order of the operation matters, is the hierarchy in the abstract syntax tree. Consider the assignemt \textit{x=x+1}. In an abstract syntax tree, the reading of the value \textit{x} and the addition of 1 is executed first. It is also at a deeper level of the tree than the assignment. If the assignment were to be replaced by a setter, the syntax tree of the reading expression can be out of sync because it is not linked to the original assignment. Therefore, it is vital to refactor parts of a code with higher depth in the syntax tree before parts with lower depths. 