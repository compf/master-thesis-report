\begingroup
\renewcommand{\cleardoublepage}{} % TODO maybe removing this and next
\renewcommand{\clearpage}{}
\chapter{Concept}\label{chapter_conception}
\endgroup
This chapter deals with the concept of the program. It provides information about the processing pipeline (\ref{sec:pipeline} and how the different parts of that pipeline can be modeled in a way that allows easy extension. 


\hfill
\section{Pipeline}\label{sec:pipeline}
In order to find and refactor data clumps automatically, a particular sequence of steps \textbf{(pipeline)} has to be respected. Most steps of this sequence must be in a specific order because they rely on information extracted in a previous step, or the quality of the results (which might be needed by subsequent steps) would worsen.

Each step of the pipeline is performed by a  \textbf{Handler}. A handler might handle one or more steps.  Each handler has information about what steps of the pipeline it supports and  a handler can be registered to a pipeline step if it supports the particular step.  

Since a service-based approach is used, a handler can be seen as a gateway to a different program or service that performs the specific functionality.  The handler and the respective service are tightly coupled. While a handler could deal with multiple services, it is not required by the design, and requires special care. Nevertheless, it recommendable to write the handler as abstract as possible.
\begin{figure}
    \centering
    \includesvg[inkscapelatex=false,scale=0.9]{figures/chapter3/solver_gateway_service.svg}
    \caption{Visualization of communication to services for one pipeline step}
    \label{fig:solver_gateway_service_overview}
\end{figure}

Figure \ref{fig:solver_gateway_service_overview} illustrates how the service-based approach works for one step:

\begin{enumerate}
    \item The program developed in this master thesis executes one step (e.g. finding a suitable class name)
    \item The gateway registered to that step is used. It is provided with the current context containing the results from previous steps.
    \item The gateway starts or employs a specific service. This service might be dependent on the current step or might be constant. The gateway knows how to start or use the service and sends all relevant information to the service (e.g. via files, network streams etc.) 
    \item The service processes the requests submitted by the gateway. For instance, ChatGPt can suggest a suitable class name. The service might be outside of the control of the developer. For instance, it might be programmed in a different programming language or the source code might not be available. At some point it will create a response and send it back to the gateway. 
    \item The gateway reads the response and processed it. It then creates a new context based on the previous context and the response from the service. 
    \item The new context is returned to the program and can be used for successive steps (e.~g. the suggested name can be used for creating a respective class) 
\end{enumerate}

Some parts of the pipeline are mandatory because the purpose of finding and refactoring data clumps would be defeated if these steps would not be executed. For instance, obtaining the source code of the project to analyze, finding data clumps, finding a suitable name, and the refactoring can be regarded as mandatory

In the following subsections, these steps will be outlined:
\subsubsection{Code obtaining}
Data clumps can only be found if access to the source code is given. This requires that the source code is available in some location and that the location is known.

In most cases, this is trivial since a developer needing to find and fix data clumps should know where the source code is located.

However, there might be more complex cases. For instance, the source code could be located on a \ac{VCS} or must be downloaded elsewhere so that the newest changes can be considered. Therefore, defining a specific step for obtaining the source code before executing the rest of the pipeline is useful. 
\subsubsection{Detecting data clumps}\label{subsec:chap3_data_clump_detection}
It is easily agreeable that data clumps can only be fixed if found previously. Therefore, one of the first steps of the pipeline must be the detection of data clumps. 

The data clump detection process itself can be further divided into sub-steps.
\subsubsection{Filtering}\label{subsub:filtering_files}
First of all, all files of the programming language need to be detected. Usually, there are specific file extensions (e.g., \textit{.java}) for a source code file, simplifying the finding of files. It is also common for software projects to have a root folder that contains all source code files. Therefore, one can recursively iterate through all files and directories of a given root folder with a matching extension. 

However, not every source code file might be analyzed for data clumps. For instance, parts of the projects must not be refactored, or a re-factorization would be too time-consuming. As outlined in chapter \ref{sec:introduction}, the cost of detecting data clumps might need to be lowered. Therefore, filtering rules might be necessary to reduce the size of the relevant files for the subsequent steps.  

\subsubsection{Extraction of AST}
Then,  the source code needs to be simplified as comments, the method bodies, and other language parts are irrelevant for detecting data clumps. This could be achieved by extracting a subset of the \ac{AST} from the source code, which includes all the relevant information to identify data clumps.
\subsubsection{Similarity detection}

The next step is finding pairs of method parameters and pairs of identical or at least similar fields. For this, the identifier and the data type can be compared. Unequal variable names might not rule out a similarity between two variables. For instance, typos may happen, or synonyms may be used. Also, the type might be different. For instance, the data type \textit{double} can be seen as a super-set of the datatype \textit{int} because every 32-bit integer can be converted to a \textit{double}.

From this, a graph can be constructed that visualizes the relationship of the several variables. A node represents a parameter or field value, while an edge exists if and only if a relationship between two variables is detected. 

\subsubsection{Data clump detection}
After similar variables have been detected, it must be determined whether a cluster of variables that are deemed similar is, in fact, a data clump. Here, the rules from section \ref{sec:data_clump_def} can be used. However, defining one's own rules can lead to better results, so flexibility is essential, too. 

\subsubsection{Data clump filtering or prioritization} \label{subsub:filtering_data_clumps}
In section \ref{subsub:filtering_files}, a filtering by files is performed. However, even if many data clumps were detected, they do not always need to be refactored. One can decide to ignore certain data clumps and refactor them later or even never. 

There can be many reasons for this approach. For instance, parts of the source code that are infrequently updated might be too risky for refactoring since any automatic tool might cause bugs or invalid code that is hard to fix because the developers are unfamiliar with this part of the source code. On the contrary, a part of the source code that is updated more frequently might be in an intermittent state, so that refactoring is currently not recommendable. Also, as noted in section  \ref{subsub:filtering_files}, if a service like ChatGPT is used, the cost of transferring many source code files with data clumps might be too high. 

In the case of field data clumps, we also have to consider the worst case, namely that at least one of the data clump fields is public. This is considered bad design as fields should always be encapsulated by getters and setters \cite{5680918}. Nevertheless, it is still prevalent \cite{5076631}.

A subsequent refactoring of these fields is more complicated than that of less accessible fields because they might be used by external programs or libraries that are beyond the control of the developer, which might induce bugs or non-compilable software. 

One can argue that refactoring should never affect the public interface since the above-mentioned issues can occur \cite{10.1145/1352678.1352681}.

On the other hand, one can argue that those changes are still significant because they can improve readability, and the source code may be in the early stages to make such breaking changes feasible. 

Therefore, filtering out certain data clumps considered not worthy of refactoring can be suggested.

\subsubsection{Name finding}\label{subsec:chap3_data_clump_name_finding}
The next step of the pipeline is to find a suitable identifier of the extracted class.

There are some criteria for this identifier. It should be a valid name in the respective programming language so that, in general, only alphanumerical characters may be chosen. It should not conflict with other identifiers as it may lead to bugs or compiler errors, and it should be readable and, as best as feasible, represent the fields of the class.

Hence, the name finding is a complex process requiring domain knowledge and creativity if the resulting name is accepted. 
Since English is the predominant language of identifiers, it will be assumed that only English identifiers will be used. 


\subsubsection{Class Extraction}\label{subsec:chap3_data_class_extraction}
After a class name has been found, a respective class can be added to the source code. This class will be used as a replacement for the data clump variables or parameters. 

In contrast to the previous step of the pipeline, this process can be automated more straightforwardly because it does not require creativity and a general template can be used, that is:
\begin{itemize}
    \item Create a private field for each variable of the data clump
    \item Create a public getter for each variable of the data clump. This getter may be named get\textit{Var} where \textit{Var} is replaced by the capitalized name of the variable. The method returns the value of the respective field
    \item Similarly, a setter set\textit{Var} may be created that has no return type, accepts a new value as a parameter, and sets the respective field. 
    \item A constructor that initializes the fields with provided or default values may be created. 
\end{itemize}

Additionally, methods for converting an object to a string or creating a hash code may be useful, but these methods are usually optional for refactoring data clumps. 


\subsubsection{Refactoring}
After extracting a class, the next step is to refactor the source code to remove the data clumps. Here, again, we need to differentiate between method parameter data clumps as this impacts the access scope. 

In case of a field data clump, the following transformations need to be applied.

\begin{enumerate}
    \item For each class in which the same field data clump is detected, the data clump variables are replaced by an object of the extracted class
    \item A getter of that object is created that has the same access level as the highest access level of the data clump items
    \item If a data clump item is read, it must be replaced by a call of the respective getter of the extracted class
    \item Similarly, if a data clump item is written to, the assignment must be replaced by a setter
\end{enumerate}



In the case of a method data clump, the general approach is similar to the case of field data clumps. An object of the extracted class replaces the data clump items, and all references to the data clump items are also updated. Here, the object of the extracted class is not a getter because it is only locally defined. Additionally, the calls to the refactored method must be updated. Instead of providing the arguments for the data clump items directly to the method, an intermediate object can be created that is filled with those values, and this object is provided to the method. 

\subsection{Validation}

At the conclusion of the refactoring process, it is not certain that everything is perfect. This applies to manual refactoring and also automatic refactoring. For instance, there might be an undetected error, the refactoring leads to ambiguous method signatures, some reference are not up-to-date etc.

As a result, it is critical to test the project after refactoring so make sure that all requirements are still fulfilled.

Many build systems like \textit{Gradle} or \textit{Maven} support simple commands to run all unit tests and build the project in a consistent manner.  Therefore, one can run such a command and check the exit code or output to determine whether the project compiles and passes unit tests.

Disadvantageously, this may only be effective if the project is correctly configured with a build system. This might not always be the case, and it is not trivial to construct an interface for a general build solution. 


\section{Context}
In section \ref{sec:pipeline}, the several steps of the data clump pipeline are outlined. However, these steps do not include how the information generated by each step is stored and how successive steps might use this information.

Therefore, a concept to store intermediate information needs to be developed. Such information might be called \textbf{context}.

The context is a storage filled throughout the data clump refactoring pipeline. For instance, it might be stored in memory or a database if the context size is too large.

Each step in the data clump refactoring pipeline can access the context and obtain such information as it may deem necessary for its use. A step may or may not update the context. For instance, if a service performs multiple step consecutively, it might not be able to store some information. To give a more concrete example, if a service can find data clumps and refactor them automatically, it might not be able to store the locations of each data clump so that respective context cannot be created. 

\begin{figure}
    \centering
    \includesvg[inkscapelatex=false,scale=0.9]{figures/chapter3/context_pipeline.svg}
    \caption{Visualization of context pipeline}
    \label{fig:context_pipeline}
\end{figure}

The context can be represented as a linked list. Each node of the list represents a part of the context. Figure \ref{fig:context_pipeline} illustrates how the context can be  represented.

Here, in the beginning, only the location of the project to analyze is stored, which is mandatory. The pipeline step executing the \textit{DataClumpDoctor} can use this context and append a new context. A name finding service like ChatGPT can then append a context with the names of the extracted classes. At the end, a service like IntelliJ can do the refactoring.

To check whether a given pipeline is indeed executable, each step handler must describe how the context is updated when the the step is executed. Each step can also specify whether a specific type of context must exist. For instance, a step may require that the names of the extracted class of each data clump must be stored in the context.  

Such context can only work if the context data is in a specific format, which can be generated by one step and might be correctly interpreted by each successive step. 

In the following, the context created or updated after each step will be explained:

\subsubsection{Code obtaining}
The context after obtaining the source code of the project to analyze is usually the path to the project after it. In most cases, the project is defined in such a manner that there is a single base directory under which all files and directories of the project are located.

Alternatively, one could store the list of all relevant file paths of the project. This requires more storage but makes it easier to filter out files. 

\subsubsection{Filtering}
After filtering the files, the context can be a new path with all files considered relevant, requiring a deletion or moving file system operation. If a list of files is used, all irrelevant file paths can be deleted, which might be faster.

\subsubsection{Extraction of AST}
\subsubsection{Data clump detection and filtering}

The format described in section \ref{sec:data_clump_format} can be used to store the detected data clump. While this format is relatively new, it contains all relevant information for storing data clump information and is extendable. However, there are other reasons for other data formats. For instance,  the data clump type context format might be too detailed, which can lead to storage or performance issues. 

\subsubsection{Name finding}
To store the determined names per found data clump, a dictionary can be used that maps a unique data clump to a name. For large instances, this could be done via a database. 

Since there could be multiple name suggestions for a given data clump, multiple names might be stored so that a correct name may be chosen later. However, this is not part of the modeled pipeline as the benefits do not outweigh the increased complexity.

\subsubsection{Class extraction}

After a class extraction step, the created class can be stored somewhere to be considered part of the project. The exact location can significantly impact the readability of the source code, as the location of files in a software project can help in understanding the project. 

For instance, the class could be where the data clump is initially found. This can lead to arbitrariness as the exact order of where and when data clumps are detected might not be predetermined. 

Alternatively, specific locations can be used to store all extracted classes. This, however, can also hinder readability as the extracted classes have no connections to the places where they are used. 

As an alternative, the complete class body could be stored instead of saving it directly to a file. This might advantageous if the class content should be formatted, refactored or otherwise modified in order to be valid. This also would deflect the responsibility of the class location from the class extractor step. 

One might also argue that choosing the extracted class location should be separate step. 
However, too many steps can be counterproductive as they increase the complexity and causes the individual steps to have too few responsibility.

Therefore, the class location is determined by the extractor which also saves the class content and stores the data in its context. 

\begin{comment}

\section{Pipeline execution example}
Listing \ref{lst:math_stuff_java} will be used as the foundation to describe a detailed approach for fixing data clumps.

Looking at the parameters \textit{x}, \textit{y}, and \textit{z}, there are multiple suggestions for a class name. The most trivial one is a simple concatenation  (e.g., \textit{XYZ} or \textit{X\_Y\_Z}). However, this name is often not suitable as it can be hard to read and to understand its meaning.

A more suitable approach requires domain knowledge. It is common knowledge that the terms \textit{x}, \textit{y}, and \textit{z} are terms used in math to describe coordinates. Using this information, a fitting name could be \textit{Coordinate}. Simple methods would not have discovered this name as artificial or human intelligence is required. 



Since a suitable class name has been found, the following class can be created using previously obtained information. 

Listing \ref{lst:coordinate_java} shows an example implementation of a coordinate class. This class contains fields, getters, and setters for all method parameters. There is also a constructor for initialization. This respects Fowler's opinion that the extracted class should not be a mere data class but a functional class that can be easily extended and modified.


\end{comment}






\hfill
