\addchap*{Abstract}
\textbf{Deutsch}
Die Softwaredokumentation ist ein essenzieller Bestandteil der heutigen Softwareentwicklung geworden. Nichtsdestotrotz leidet die Qualität der Dokumentation häufig und viele Entwickler sind nicht motiviert genug, um eine gute Dokumentation zu schreiben. Das Ziel dieser Arbeit ist es, ein Tool zu entwickeln, dass exemplarisch die Dokumentationsqualität in Java-Programmen analysiert und mittels verschiedener Metriken (Anteil dokumentierter Komponenten an allen Komponenten, Flesch-Score, Kohärenz und Nichterwähnung von Randfällen) bewertet. Dieses Tool ist in GitHub Actions eingebunden, um den Entwickler bei einer sehr schlechten Dokumentationsqualität zu warnen und gegebenenfalls Mergevorgänge zu verhindern.

%\linebreak
\bigskip

\noindent
%\bigskip
\textbf{English} 
The software documentation has become an integral part of software development. Nevertheless, the quality of the documentation is often poor and developers are often not motivated to write good documentation. The goal of this thesis is to develop a tool that can analyze the documentation quality of Java applications by applying different metrics (percentage of documented components in all components, Flesch score, coherence, not mentioning the handling of edge cases). This tool will be integrated in GitHub Actions to warn the developer about poor software documentation quality and to prevent a merge if the quality becomes too poor.  

%TODO Bis jetzt nur Osi Abstract, evtl. etwas ausführlicher für Masterarbeit