\addchap*{Abstract}
\textbf{Deutsch}

Code-Smells (z.~dt. übelriechender Code ) gelten als zuverlässiges Anzeichen für Quellcode mit Qualitätsprobleme, was zu höheren Wartungskosten führt. Refaktorisierungen von Code-Smells können diese Probleme lösen, haben aber in der Praxis nur eine geringe Priorität. Eine Lösung können automatisierte Refaktorisierungen sein, die den Entwickler bei der Behebung unterstützen. Der Fokus dieser Masterarbeit wird auf das Code-Smell „data clump“ (z.~dt. Datenklumpen) liegen, bei dem Variablen im Quellcode dupliziert werden. Bislang gibt es noch keine automatisierten Werkzeuge zur Behebung dieses Code-Smells. Mithilfe von Sprachmodellen wie ChatGPT, die  seit einiger Zeit große Aufmerksamkeit erregt haben,  wird ein modulares Werkzeug entwickelt, das die Detektion und Behebung von Data-Clumps automatisch durchführen kann.  Die Eignung von ChatGPT für diese Aufgabe wird  anschließend mittels einer Umfrage auf GitHub und anderen Experimenten untersucht. 
%\linebreak
\bigskip

\noindent
%\bigskip
\textbf{English} 
The existence of code smells is a reliable indicator for code quality issues which often induces higher maintenance costs. Refactoring code smells is an effective way to improve the quality of the code, but is not the priority of many developers.  Therefore, automatic refactoring tools can help to support developers to fix code smells. This master thesis focuses on one particular code  smell named data clumps, which is the duplication of variables across the code. No automatic refactoring tool for this code smell exists. Employing the capabilities of large language models such as ChatGPT that  recently have gained widespread attention, a modular tool is developed  that automatically detects and fixes data clumps. The thesis evaluates whether ChatGPT is suitable for this task via a GitHub pull request survey and other experiments. 

