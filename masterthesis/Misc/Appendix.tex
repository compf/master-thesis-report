
\renewcommand\appendixpagename{Appendix}
\begin{appendices}
\chapter{Data Clump Type Context}
\label{sec:data_clump_format}
An example of the Data clump type context can be seen in listing \ref{lst:data_clump_type_context_example}. Only a subset of the format will be discussed here for space reasons.

  \begin{figure} [htbp!]
			\lstinputlisting
			[caption={ Resulting Coordinate class},
			label={lst:data_clump_type_context_example},
			captionpos=b,language=java, basicstyle=\footnotesize, tabsize=2, showstringspaces=false,  numbers=left]
			{figures/chapter2/data_clump_type_context_example.json}
		\end{figure}

The format consists of three layers. In the outer layer left out here, general project information is defined: The programming language or the project's location. Also the number of methods, classes, and the number of detected data clumps can be obtained in this general part. 

In the next layer, each detected data clump is mapped with a unique key (l. 4).

The detected data clumps are described as a link between two files. These files might be identical if the data clumps are located in the same file. Here, one must differentiate between the \enquote{from-part}, and the \enquote{to-part} representing the two nodes in the data clump graph. For instance, \enquote{from\_file\_path} indicates the location of one part of the data clump, while \enquote{to\_file\_path} provides the path to the opposite end. 

A similar principle is applied to the classes in which a data clump is located. Because a class name might not be unique, not only the names of the two classes but also unique identifiers of those classes are provided (l. 9-10 and 14-15).

The information about the methods of the data clump (l. 11, 12, 16, 17) is optional. If one part of the data clump is a field-to-field data clump, no method is involved, so the respective part would be \textit{null}.

The field \enquote{probability} can be used by probabilistic data clump detection tools to indicate the probability that the detected data clump is indeed a data clump. For purposes of this master thesis, it will be ignored. 

In lines 19-42, each variable that is part of the data clump is described. Here, only one variable is listed, although for a data clump there must be at least three variables. As for the data clump itself, the individual variables are separated into a \enquote{from-part} and a \enquote{to-part}. The former is implicitly defined (l.~20-25, 38-42), while the latter has a specifically named sub object \enquote{to\_variable} (l. 27-37). For each variable, the name (l. 22 and l. 28)) and the data type (l.~23 and l. 29) are provided. As for the general data clump, a probability is given whether the variable is indeed part of a data clump (l. 24). As mentioned above, this information will be ignored.  Additionally, modifiers like \enquote{private}, \enquote{final} etc. are also stored. 

In order to find the variable in the source code, detailed location information is needed. Parts of the location information are, the line number (l.~32 and l.~39) and the column number (l.~33 and l.~40). To avoid any ambiguity, the end position of the variable is saved, too. All numbers are one-index-based, meaning that the first line is one and the first column is also one. 


\chapter{Pull Request Text}


Title:  Refactored data clumps with the help of LLMs (research project)


Hello maintainers,

I am conducting a master thesis project focused on enhancing code quality through automated refactoring of data clumps, assisted by Large Language Models (LLMs).\newline


\textbf{Data Clump Definition}
\newline

A data clump exists if
\begin{enumerate}
    \item  two methods (in the same or in different classes) have at least 3 common parameters and one of those methods does not override the other,  or
    \item  At least three fields in a class are common with the parameters of a method (in the same or in a different class), or
    \item  Two different classes have at least three common fields
\end{enumerate}

See also the following UML diagram as an example
\begin{figure}[htpb!]
    \centering
    \includesvg[width=0.75\columnwidth]{figures/appendix/data_clump_explained.svg}
    \caption{Example Data Clump}
    \label{fig:enter-label}
\end{figure}
    



I believe these refactoring can contribute to the project by reducing complexity and enhancing readability of your source code.

Pursuant to the EU AI Act, I fully disclose the use of LLMs in generating these refactorings, emphasizing that all changes have undergone human review for quality assurance. 


Even if you decide not to integrate my changes to your codebase (which is perfectly fine), I ask you to fill out a feedback survey, which will be scientifically evaluated to determine the acceptance of AI-supported refactorings. You can find the feedback survey under \url{https://campus.lamapoll.de/Data-clump-refactoring/en}


Thank you for considering my contribution. I look forward to your feedback. If you have any other questions or comments, feel free to write a comment, or email me under tschoemaker@uni-osnabrueck.de.
\newline

Best regards,\newline
Timo Schoemaker \newline
Department of Computer Science \newline
University of Osnabrück \newline



\end{appendices}
	
