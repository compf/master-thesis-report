
In this chapter, the full-scale evaluation is discussed.
\section{Structure of the evaluation}

The evaluation is separated into two parts.

In the first par, the tool was used on multiple public GitHub repositories to measure the acceptance of \ac{LLM}-assisted refactoring.

In the second part different applications of the tool were tested and compared with manual approaches to determine the usefulness and challenges of the tool while detecting and refactoring data clumps.





\section{GitHub project selection}\label{sec:github_projects}

To facilitate the evaluation, GitHub projects must be chosen which will be analyzed in the two parts of the evaluation. 


The projects are selected from the trending page of GitHub. This page list GitHub projects that have gained attention in a specific time period. For instance, they were forked comparatively often or received more stars in comparison, but the exact criteria is difficult to determine. Each project was tested on the following criteria
\begin{enumerate}
    \item Whether the project contains at least  10,000 \ac{LOC} of Java
        \item Whether the project  has at least 100 stars

\end{enumerate}

With the first two criteria,  only larger projects are considered. This increases heuristically the chance of having a higher number of data clumps, and also the chance of getting more developers to respond to the survey as larger projects tend to be maintained by more people. The third criteria also influences the number of contributors and their willingness to work on the project. Only if pull request are frequently considered and closed (which does not necessarily mean that they are merged), the project is considered active enough. 


\section{Pull request evaluation}
In this section, the first evaluation about pull request related to data clump refactoring is discussed. 

\subsection{Goal of the evaluation}

To ascertain the quality of refactoring, human feedback is essential. While there are metrics to evaluate whether a given refactoring is useful, the metrics do not always align with the viewpoints of developers. 

 For instance, while data clumps can be detected with the DataClumpDoctor, each detected data clump might not be a data clump for every developer. For instance, developers may disagree about the required number of data clump items (three in this master thesis). Additionally, developers who have knowledge of the structure of a project can better determine whether a group of variables  constitutes a data clump. In some cases, they have written the method or class themselves and have introduced the data clump on purpose because they could not find a proper name for the extracted class, or the disadvantages discussed in section \ref{sec:data_clump_not_refactor} outweigh the advantages of introducing a code smell. 

Even if developers agree that a detected data clump constitutes a data clump  in their opinion, there could be arguments against refactoring them. These reasons include the points  discussed in section \ref{sec:data_clump_not_refactor}.Additionally, there are more grounds that are not specific to data clumps but code smells in general. For instance, the location of the code smell could be currently under heavy editing so that it might not be certain whether the methods or classes will exist in the long-term. On the other hand, parts of the source code that have been untouched for longer times, could be too risky to refactor as the knowledge on how the source code behaves or is structured is limited, so that developers would not refactor them even if it might make sense. 

As a result, asking the opinion and ideas of developers on whether data clump refactoring is justified and can be performed by a \ac{LLM} is the basis of this evaluation .




\subsection{Criteria for selecting data clumps}

For each selected project, one data clump was chosen. Two selection approaches were used. 
In the first approach, a data clump was chosen based on a weighted combination of metrics to identify an \enquote{important} data clump. These metrics are:

\begin{itemize}
    \item The number of occurrences of data clump. For instance, if there is a data clump with the variables \textit{x}, \textit{y}, and  \textit{z}, it is counted how many methods have these parameters and how many classes have these fields

     \item The number of data clump items. For instance, the \enquote{xyz}-data-clump has three data clump items.
     
    \item The number of affected files. Every file that is affected by a data clump. This includes the location of the data clump and also the files where methods and fields that are part of the data clump are references because these files must also be changed if the data clump is refactored. 
    
\end{itemize}

From these scores, for each data clump a weighted sum was calculated. For the first two metrics, either the weight is 100 or 1. In these metrics, data clumps that occur more often or are larger are scored better so that they are more likely to be refactored because the code size can be reduced more strongly.  For the last metric, the weights were -100 or 0. A zero or negative weight was chosen because a large number of affected files can be obstacle for  refactoring as many areas of the codebase may need to be changed. If an \ac{LLM} performs the refactoring, all those files must be transmitted to the model which increases costs and resource usage. Even if \ac{LLM} is involved only marginally, it is more likely that the less files are changed the more contributors on GitHub are willing to give feedback.

For each combination of the weights described above, the five most-scored data clumps were manually reviewed. After that, one particular data clumps was  chosen from all combination. The criteria for selecting this data clump was more subjective as it is difficult to determine which data clump would be  the most relevant to refactor. The following considerations influence the filtering process.

\begin{itemize}
    \item Avoid data clumps that affect abstract classes or interfaces as they should not be changed.
    \item Avoid data clumps that only affect source code for unit testing. While refactoring source code for tests is very important too, data clumps in the main code are regarded more important for the purpose of this thesis.
    \item Whether the combination of the parameters would make logically sense. for instance, are the fields or parameters in similar domains or are they used together. 
    \item In case of fields, potentials issues that occur if dependency injection is used because moving these fields require special attention that is outside the scope of this master thesis
\end{itemize}
All these criteria are more guidelines than strict requirement to allow flexibility 
After considering all of these criteria, one final data clump was selected.
\subsection{Assignment to a category}

After a data clump is selected, the next step is to assign the project to a category to determine the extent \acs{LLM} are used to find and refactor data clumps. 

Two categories are to be distinguished which are explained in the following subsections:

\subsubsection{Full refactoring by ChatGPT}
In this method, ChatGPT performs the refactoring completely. Because transmitting whole GitHub projects would infeasible, the DataClumpDoctor was used to detect the previously selected data clump and obtain all lines of interests. A neighborhood of 5 was used so that 5 lines below and 5 lines above each location of interest was transmitted. 

Then, ChatGPT is instructed to refactor all data clumps in the provided lines of interest. This instruction was repeated at least ten times, in each time the context of ChatGPT was cleared so it didn't know its previous answers. 

From these ten proposal, one proposal was chosen that describes refactoring data clumps most accurately. For instance, the extracted class is valid, most usages of the data clump items are updated and all method signatures are refactored (if applicable). Generating multiple proposals in necessary because not every proposal will be correct.

\subsubsection{Refactoring by IntelliJ}

The second approach for refactoring was via IntelliJ. In this case, ChatGPT only suggest as suitable name for the extracted class, but is otherwise not involved in the refactoring. Instead, IntelliJ, performs all refactoring in the manner described in section  \ref{sec:intellij_refactoring} is applied. This results in a very consistent refactoring without any creativity. Hence, this refactoring needs only to be executed once and the first proposal can be selected immediately. 

After selecting a proposal, the proposal is applied and saved on a separate branch.

Afterwards, the proposal might not be fully correct. For instance, there might none-updated method calls, missing semicolons etc. An additional problem occurs if codestyle tools like SpotBugs or Checkstyle are employed. If the refactoring by the \ac{LLM} does not conform to the required codestyle, the code might not compile because the developers of the project force a certain style. All these are issues that can still arise and require additional handling. Therefore, a manual correction step is performed. The project is  manually changed in such a way that it fully compiles. However, no creative refactoring is performed. For instance, if one part of the source code was not refactored, it was refactored like another part regardless of whether another refactoring might have make more sense. This reduces human intervention to a minimum and ensures that the creative part of the refactoring is done by the \ac{LLM}. 

As soon as this manual refactoring finished and the program compiles, the changes were squashed into one commit and a pull request was created in the respective repository. In this pull request, the maintainers were described the purpose of this pull request and  the definition of data clumps used in this master thesis. They were asked to give feedback by filling out a feedback form or by giving feedback via GitHub comments under the respective pull request. It was explicitly stressed that rejecting the pull request would not be perceived negatively. 

The feedback by the form and comments were collected and evaluated as described in subsection \ref{sec:feedback_survey}

\subsection{Feedback survey}\label{sec:feedback_survey}

The feedback consists of two parts.
\subsubsection{GitHub comments}
One  natural way of providing feedback over Github is via comments. These commeent can be review comments that address specific parts of the code or general comments unrelated to any code that addresses the pull request as a whole. \cite{10.1145/3597208}. Therefore it is an important source of determining the acceptance of the proposed refactoring. However, since the comments are natural texts, obtaining meaningful results is more challenging. 

\subsubsection{Likert scale}
The likert scale is a common method in surveys. It consists of statements that claim a certain fact and the respondents have to express their opinion to each statement by choosing from one discrete attitude category. \cite{edmondson2005likert}

For instance, a statement might be \enquote{refactoring data clumps is useful}, and the list of available attitudes can be
\begin{itemize}
    \item Strongly agree
    \item Agree
    \item Neutral
    \item Disagree
    \item Strongly disagree
\end{itemize}

Using the method evaluating the variance of opinion can be eased as the possible answers are discreet and easy to map on a numerical scale.(e.~g. 0 for strongly disagree to 5 for strongly agree).


For this master thesis, the survey plattform \textit{lamapoll} \cite{lamapoll} was used.  The following statements should be addressed by the respondents:
\begin{enumerate}
\item Data clumps are a code smell that should be fixed, 
\item Using LLMs in software development can be helpful to improve code quality,
\item The proposed refactoring maintains or improves the quality of the code.,
\item The proposed refactoring has  adequately identified and preserved the original functionality and intent of the code, 
\item The name of the new extracted class(es), fields and methods are well-chosen
 \item The location of the extracted class(es) are well-chosen,
 \item For how long have you been contributing to this project?,
\item For how long have you been a developer in Java ?, 
\item Please input the URL of the GitHub project from where you got to this survey

\end{enumerate}


The statements 7--9 are actually questions not using the likert scale. Question 7 attempts to assess the experience of the developer about this project. Question 8 similarly is aimed to obtain the programming experience of the respondent in the Java language.

The purpose of question 9 is to create a connection between the survey and the pull request of the project as the survey platform does not allow such a link easily. 


\section{Data clump filtering by model}

In this evaluation, it is tested whether the  \ac{LLM} is suitable for data clump filtering. As outlined in section \ref{sec:data_clump_filtering}, there are multiple approaches for filtering data clumps, which can already be implemented manually. 

The question here is whether the \ac{LLM} uses novel filtering approaches or simply relies on the metric discussed in section \ref{sec:data_clump_filtering}. In the latter case, it is more useful to use the manual algorithm because they are reliable and do not incur the costs associated with Large Language Models. 

In order to facilitate this experiment, the data clumps are evaluated on 3 metrics (size, occurrence, affected files. Then the ten most important data clumps are randomly shuffled and presented to the \ac{LLM} which chooses one data clump. 

The following metrics are used to evaluate this experiment:
\begin{itemize}
    \item The rank of the chosen data clump with regard to each of the three metrics mentioned
    \item the weighted combination of the three metrics
    \item the easiness of refactoring this data clump
\end{itemize}


\section{Data clump refactoring}

Evaluating how an \ac{LLM} performs data clump refactoring is another method to assess the suitability of the models for use in the pipeline. Here, especially the creativity is important. If a model merely refactors similary to a manual tool (e.~g. IntelliJ), it has less use.

If however, the \ac{LLM} extracts more functionality by creating new methods ore solves the data clump in other ways, the advantages of the model become more obvious.


The following metrics are used for evaluating this tests:

\begin{itemize}
    \item The number of attempts until the refactored program compiles
    \item whether the extracted class are more than mere data classes
    \item the existence of documentation/comments
\end{itemize}

\section{Data clump detection}

While tools for detecting data clump are well implemented in the main program, using an \ac{LLM} can provide opportunities. For instance, data clumps can be detected if two variables are synonyms, also consideration like annotations, modifiers, or associated comments can be a possible indicator to assess whether a data clump does in fact exists. In this regard, data clump detection and prioritization can be very similar. 

The following metrics are used for evaluating this tests:

\begin{itemize}
    \item The sensitivity and specificity compared to the \textit{DataClumpDoctor}
    \item the completeness and correctness of the metadata (line numbers, modifiers, etc.)
    \item the number of data clumps that have similar but not equal name)
\end{itemize}