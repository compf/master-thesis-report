\section{Implementation of handlers} \label{sec:step_impl}

\subsection{Data Clump Doctor}

The main tool to detect  data clumps is the \textit{DataClumpDoctor}. It can be directly integrated into the main program as it is written in TypeScript.

The \textit{DataClumpDoctor} works by generating an \ac{AST} of every class in the project to analyze, so that the syntax tree of each class is stored in a different file.  This task can be performed by PMD.

After that, the \textit{DataClumpDoctor} loads the \ac{AST} files and detects the data clump by comparing the identifier and types of variables that might constitute a data clump, thereby requiring strict equality. These detected data clumps are saved in a separate file and can be re-used in the pipeline. 


\subsection{Name finding}
Finding a suitable name for the soon-to-be-extracted class is the next step  that must be implemented. Two approaches are chosen.

The first approach is the classic approach that has been used before the advent of \acs{LLM}. By concatenating all field names of the extracted class, a valid class name can be generated. For instance, a class with the fields \textit{x}, \textit{y}, m\textit{z} could be named \textit{XYZ}. In this example, the field names are capitalized and directly joined without any separator, however, other options might be better dependent on the project's style guidelines. 

The advantage of this manual method is that the generated names are very unique and with a high probability will not conflict with other names because they are very artificial. Nevertheless, they lack creativity and in most cases, a developer will need to choose a better name in order to improve readability. 


On the other hand,a suitable name for the extracted class of a data clump can be chosen by using the creativity of \acs{LLM}.  For this the model has to know the names of the variables of the data clump. It is also useful to include the qualified type of each data clump item because this type contains additional information to generate a more suitable name. For instance, the qualified name could contain the name of the project to analyze, the location of the type, and the range of possible values that the variable can have.

While using \acs{LLM} for name finding, excessive name caching can be useful to save costs and resources. The same combination of data clump items will probably result in the same or at least a similar class name. In order to prevent the extraction of classes with similar purposes, it is therefore useful to only ask the model if the combination of data clump items has not occurred before. 

A related issue to name finding is the location of the extracted class. Here, one has to differentiate  between data clumps existing in a single class and data clumps connecting multiple classes. In the former case, the location of the extracted class can be an already existing file, namely the file of the data clump  while in the latter case, a new class should be created.

Using an existing class can be problematic because the succeeding step (class extraction) must be mindful that an existing class is used as it could override the file. Additionally, integrating the new class into the existing class can be challenging. Should the new class be an inner class of some class? In Java, should the inner class be static?  Creating an inner class require strategic syntactical modification of the source code which would require specialized handlers. As a result, they are not implemented in this master thesis, but could be. 

On the other hand, using a completely new file is easier to implement. However, one has to choose the path of the extracted class. For instance, a data clump might be spread over \textit{n} files each of which is in a separate directory, thereby creating a theoretical possibility of \textit{n} candidates as the output directory. 
One has also be mindful about any name conflicts that might occur. If the suggested name already exists, it will lead to conflicts.


\subsection{Class extraction}

Generating the actual class is a task that can be done manually if the names and type of the fields are known and the name of the class is known. However, the specific syntax depends on the programming language so that each programming language must have its own class extractor.

Other issues should also be taken into consideration. For instance, whether the class body at first contains the declaration of fields, then the getters, setters, and then the constructor, or another order is better depends on the project's style. For instance, the constructor could be declared directly after the fields.

Additionally, not every part of the extracted class might be mandatory. For instance, some fields will never be modified so that a setter would be unnecessary. Furthermore, the extracted class should conform to code styles guidelines. 

Two approaches are implemented. In the manual method, the class is extracted using simple text manipulation. The order of the members of the class can be configured.  However it would be challenging to determine whether a setter for a field is needed. Hence, in the manual approach all getters and setters are generated.

Here also, the creativity of an \ac{LLM} can be useful. Given a suitable context, an \ac{LLM} can return the source code of the extracted class without being restricted to a template as the manual approach is. For instabce, instead of generating a class, the model could create a Java record which is a simplified version of a class. 

\subsection{Reference Finding}


\subsubsection{Via \ac{LSP}}
In order to refactor data clumps, all relevant locations of interests (references) must be identified. If the IntelliJ plugin discussed in the next subsection is used for refactoring, this step can sometimes be left out. However, since this plugin has some issues and the refactoring might be performed by an \ac{LLM}, it is beneficial to implement a service for reference finding. 

The \ac{LSP} is one method to find these references because it works reliable and is also available for other programming languages.

Four kinds of references are relevant for data clump refactoring.

\begin{enumerate}
     \item A  variable (field or method parameter)  is declared
    \item A variable is read from or assigned to
   
    \item A method is declared or overridden
    \item A method is called
\end{enumerate}
To facilitate the use of \ac{LSP}, the server is started and initialized. After that, a request to obtain references for each (filtered) data clump is sent to the server.
Any reply by the \ac{LSP} is intertwined with the associated data clump thereby creating the reference context.

 


\subsection{Refactoring by IntelliJ} \label{sec:intellij_refactoring}

For  manually refactoring data clumps, the  \ac{API} from IntelliJ mentioned in section \ref{sec:psi} can be used because it allows for easy modification of the source code that does not result in faulty code. 


\begin{comment}The reasons for this are difficult to determine and the documentation is scarce, so the the PSI approach seems to be only suitable for projects created via IntelliJ or correctly initialized by IntelliJ with the required meta data. Gradle and maven projects are therefore not suitable for the full refactoring step. 
\end{comment}


The basic approach for refactoring is based on modifying references. A  reference is an information about the usage of an element. Element in this context could mean a class, a method, or a variable.
In many cases , IntelliJ can find all references automatically and categorize them.

However, this does not always happen. A project can be loaded wrongly so that  finding references of a method or field can lead to undefined behavior. Sometimes all references are correctly found, sometimes only a subset of the references are found, and sometimes no even no references are detected. This could be explained by invalid caches or unsupported projects although it is difficult to determine whether there are reference errors as no ground truth exists.

One solution for these issues is to open the project in IntelliJ at least once  on the same computer running the tool, so that all references can be loaded. Therefore, using IntelliJ as the main \ac{IDE} of the project to analyze will maximize the chance of finding all references



If this approach does not work, the context can be utilized. If a previous created a context that contains reference information, the IntelliJ plugin can use this context to perform the refactoring. This approach represents the pipeline idea that many tools need to work together to achieve a common goal.


Whether such manual search is faster than the combined use of \ac{LSP} and IntelliJ is difficult to determine. On the one hand, can external services be faster because they are implemented better (i.~e. more sophisticated algorithms). On the other hand, starting two services leads to more overhead. 



Regardless of how the reference issue is solved, IntelliJ can now perform the refactoring. Depending of the category of a reference , IntelliJ needs to perform distinct step. 
\begin{enumerate}
    \item In case of a declared field, the field can be deleted because it is part of a field data clump. Now it can be determined whether the class contains already the new field that replaces the fields of the data clump.
    \item If a method is declared, IntelliJ can modify the signature of the method. This can be the original method or an overridden one. IntelliJ needs to remove the parameters that are part of the data clump and add a new method parameter which replaces the method parameters of the data clump. 
    \item If a variable is used it can be replaced by a getter or setter call. For instance if the variable \textit{var} is read, and the name of the extracted class is \textit{Object}, any reading of the variable can be replaced by  \textit{object.getVar()} where \textit{object} is a variable of type \textit{Object} and \textit{getVar} is the getter of \textit{var}. Similarly, an assignment can be replaces by the setter.
    \item If a method is used, several substeps are needed.
    \begin{enumerate}
        \item First, for each argument provided in a method call it is determined whether the argument is connected to a data clump variable (i.~e. it provides a value to a parameter that is part of a data clump) 
        \item The position of those arguments is stored and a reference to the argument is stored for further processing.
        \item Since the extracted class is known and existing, a matching constructor is determined to support all arguments to the data clump variables of the method call. 
        \item For each argument to a data clump variable, the argument is inserted into the matching position of the constructor, and the argument is deleted in the original method call. 
        \item the constructor call is added at the position of the method  where the parameter of the extracted class is expected. 
        
    \end{enumerate}
    
\end{enumerate}

This approach requires some coordination. For instance, the order of operation is important. Method and field declarations must be updated first because they are needed to perform the refactoring of variable usages and method calls successfully. 

Another aspect, where the order of the operation matters, is the hierarchy in the abstract syntax tree. Consider the assignemt \textit{x=x+1}. In an abstract syntax tree, the reading of the value \textit{x} and the addition of 1 is executed first. It is also at a deeper level of the tree than the assignment. If the assignment were to be replaced by a setter, the syntax tree of the reading expression can be out of sync because it is not linked to the original assignment. Therefore, it is vital to refactor parts of a code with higher depth in the syntax tree before parts with lower depths. 
