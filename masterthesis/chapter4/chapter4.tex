\chapter{Implementation}\label{chapter:implementation}


In this chapter the design choices and considerations discussed in chapter \ref{chapter:conception} are expanded to develop a fully functional automatic tool. 

In the beginning, section  \ref{sec:step_impl} discusses how the traditional tools are integrated into the pipeline so that they can participate in refactoring data clumps. 

Afterwards, section \ref{sec:implementation_tools} outlines how a \ac{LLM} can work with source code. To facilitate this, the model must somehow gain access to some representation of the source code. Then, the reverse direction is discussed as the output of the model must somehow be integrated into the context, or directly as refactoring into the source code. 

Because of the limitations of \acp{LLM}, filtering is an important part of the pipeline. Hence, the architecture of the filtering process and examples of various filters are explained in section \ref{sec:data_clump_filtering}.

Lastly, section \ref{sec:config} deals with the configuration of the tool. This includes how a step is linked to a handler via a configuration file and  how other objects like the \ac{LLM} to use are set up. Afterwards, there is  a brief overview of how the tool can be started and what steps happen in the background.


\section{Implementation of handlers} \label{sec:step_impl}
In this section, the implementation of the handlers is discussed. These handlers do not or scarcely use the features by \ac{LLM}. 
\subsection{DataClumpDoctor}

The main tool to detect  data clumps is the \textit{DataClumpDoctor}. As this program is written using the NodeJS platform it can be directly integrated into the main program blurring distinction between handler and service.

The \textit{DataClumpDoctor} performs both the \ac{AST} generation step and the data clump detection step although these two steps are internally separated.

If a \ac{AST} context is already present, it is re-used. Otherwise, the \ac{AST} context is generated. Here, the file filtering inclusion and exclusion rules must be applied as only the generated \ac{AST} files form the foundation for the \textit{DataClumpDoctor} to detect data clumps. Files  not represented by an \ac{AST} file are ignored. Since the detector internally uses \textit{PMD}, inclusion and exclusion rules can be set by creating a \enquote{custom-java-ruleset.xml} file. This file is located inside the Java-Archive file used by the detector and must be updated whenever \ac{AST} files are generated to apply the current filtering rules. 

After that, the \textit{DataClumpDoctor} loads the \ac{AST} files and detects the data clumps by comparing the identifiers and types of variables that might constitute a data clump, thereby requiring strict equality. These detected data clumps are saved in a separate file and can be re-used in the pipeline. 

One issue with the \textit{DataClumpDoctor} is its handling of large projects. While a concrete \ac{LOC} number cannot be given, at some points analyzing large projects becomes infeasible leading to crashes or prolonged running times. This can be explained by considering that  the number of detected data clumps grows quadratically with the actual number of data clump holders. Due to the structure of the reporting format as discussed in appendix \ref{app:data_clump_format}, there is a significant amount of repetition. For sufficiently large projects, this can be a problem it the data clump information needs to be serialized as NodeJS is not optimized for these tasks. While changing the format is one solution to this problem, it can be argued that using previous file filters is more suitable as the data clump detection results from large projects are difficult to evaluate. 

\subsection{Name finding}
Finding a suitable name for the soon-to-be-extracted class is the next step  that must be implemented. Two approaches are chosen.

The first approach is the classic approach that was used before the advent of \acs{LLM}. By concatenating all field names of the extracted class, a valid class name can be generated. For instance, a class with the fields \textit{x}, \textit{y}, and \textit{z} could be named \textit{XYZ}. In this example, the field names are capitalized and directly joined without any separator, however, other options might be better dependent on the project's style guidelines. 

The advantage of this manual method is that the generated names are very unique and with a high probability will not conflict with other names because they are very artificial. Nevertheless, they lack creativity and in most cases a developer will need to choose a better name in order to improve readability. 


On the other hand, a suitable name for the extracted class of a data clump can be chosen by using the creativity of \acs{LLM}.  For this the model has to know the names of the variables of the data clump. It is also useful to include the qualified type of each data clump item because this type contains additional information to generate a more suitable name. For instance, the qualified name could contain the name of the project to analyze, the location of the type,     and the range of possible values that the variable can have.

While using \acs{LLM} for name finding, excessive name caching can be useful to save costs and resources. The same combination of data clump items will probably result in the same class name. In order to prevent the extraction of classes with similar purposes, it is therefore useful to only ask the model if the types-names identifier has not occurred before. 

A related issue to name finding is the location of the extracted class. Here, one has to differentiate  between data clumps existing in a single file and data clumps connecting multiple files. In the former case, the location of the extracted class can be an already existing file, namely the file of the data clump  while in the latter case, a new class should be created.

Using an existing class can be problematic because the succeeding step (class extraction) must be mindful that an existing class is used as it could override the file. Additionally, integrating the new class into the existing class can be challenging. Should the new class be an inner class of some class? In Java, should the inner class be static?  Creating an inner class requires strategic syntactical modification of the source code which would require specialized handlers. As a result, they are not implemented in this master thesis, but could be. 

On the other hand, using a completely new file is easier to implement. However, one has to choose the path of the extracted class. For instance, a data clump might be spread over \textit{n} files each of which is in a separate directory, thereby creating a theoretical possibility of \textit{n} candidates as the output directory. 
One also has to be mindful about any name conflicts that might occur. If the suggested name already exists, it will lead to conflicts.


\subsection{Class extraction}

Generating the actual class is a task that can be done manually if the names and types of the fields are known, and the name of the class is known. However, the specific syntax depends on the programming language so that each programming language must have its own class extractor.

Other issues should also be taken into consideration. For instance, whether the class body at first contains the declaration of fields, then the getters, setters, and then the constructor, or another order is better depends on the project's style. For instance, the constructor could be declared directly after the fields.

Additionally, not every part of the extracted class might be mandatory. For instance, some fields will never be modified so that a setter would be unnecessary. Furthermore, the extracted class should conform to code styles guidelines. 

Two approaches are implemented. In the manual method, the class is extracted using simple text manipulation. The order of the members of the class can be configured.  However, it would be challenging to determine whether a setter for a field is needed. Hence, in the manual approach all getters and setters are generated.

Here also, the creativity of an \ac{LLM} can be useful. Given a suitable context, an \ac{LLM} can return the source code of the extracted class without being restricted to a template as the manual approach is. For instance, instead of generating a class, the model could create a Java record which is a simplified version of a class. 

\subsection{Reference finding}

Other relevant handlers deal with the finding of references. Only if all references are correctly resolved, can the data clump refactoring proceed smoothly. 

\subsubsection{Via \ac{LSP}}
In order to refactor data clumps, all relevant locations of interests (references) must be identified. If the IntelliJ plugin is used for refactoring, this step can sometimes be left out. However, since this plugin has some issues and the refactoring might be performed by an \ac{LLM}, it is beneficial to implement a service for reference finding. 

The \ac{LSP} is one method to find these references because it works reliable and is also available for other programming languages.

Four kinds of references are relevant for data clump refactoring.

\begin{enumerate}
     \item A  variable (field or method parameter)  is declared.
    \item A variable is read from or assigned to.
   
    \item A method is declared or overridden.
    \item A method is called.
\end{enumerate}
To facilitate the use of \ac{LSP}, the server is started and initialized. After that, a request to obtain references for each (filtered) data clump is sent to the server.
Any reply by the \ac{LSP} is intertwined with the associated data clump thereby creating the reference context.

 
\subsubsection{Primitive reference finding}
This handler works with all programming languages as it does not parse the source code at all. Instead, it searches for the identifiers that are part of a data clump (e.~g. method name, variable names). 

This simple approach has two issues. First of all, one must determine where to search. Searching the whole project will require a significant amount of time. Alternatively, only the files containing the data clump or the respective folder can be traversed.

Moreover, this method will lead to more false positives because it disregards the scope of variables. For instance, searching for the variable name \textit{foo} will also match local variables named \textit{foo}. Therefore, a match might not be part of a data clump, and refactoring this match can lead to more harm than good. 

This handler works best if combined with an \ac{LLM}. The model can decide for each reference whether it is truly relevant for data clump refactoring purposes. It might also propose a more sophisticated refactoring idea by using these non-data-clump references. 

\subsection{Refactoring by IntelliJ} \label{sec:intellij_refactoring}

For  manually refactoring data clumps, the  \ac{API} from IntelliJ mentioned in section \ref{sec:psi} can be used because it allows for easy modification of the source code that does not result in faulty code. 





The basic approach for refactoring is based on modifying references. A  reference is  information about the usage of an element. An element in this context could mean a class, a method, or a variable.
In many cases, IntelliJ can find all references automatically and categorize them.

However, this does not always happen. A project can be loaded wrongly so that  finding references of a method or field can lead to undefined behavior. Sometimes all references are correctly found, sometimes only a subset of the references are found, and sometimes no references are detected. This could be explained by invalid caches or unsupported projects.

The impact of this problem is especially significant because a consistent way to check  whether a project has successfully loaded is not available. This means that the reference finding via \ac{PSI} can be highly unreliable. 

One solution for these issues is to open the project in IntelliJ at least once  on the same computer running the tool, so that all references can be loaded. Therefore, using IntelliJ as the main \ac{IDE} of the project to analyze will maximize the chance of finding all references.



If this approach does not work, the context can be utilized. If a previous handler created a context that contains reference information (for instance using the \ac{LSP}), the IntelliJ plugin can use this context to perform the refactoring. This approach represents the pipeline idea that many tools need to work together to achieve a common goal.


Whether such manual search is faster than the combined use of \ac{LSP} and IntelliJ is difficult to determine. On the one hand,  external services can be faster because they are implemented better (i.~e. more sophisticated algorithms). On the other hand, starting two services leads to more overhead. 



Regardless of how the reference issue is solved, IntelliJ can now perform the refactoring. Depending of the category of a reference, IntelliJ needs to perform distinct steps. 
\begin{enumerate}
    \item In the case of a declared field, the field can be deleted because it is part of a field data clump. Now it can be determined whether the class already contains the new field that replaces the fields of the data clump.
    \item If the reference is a method declaration, IntelliJ can modify the signature of the method. This can be the original method or an overridden one. IntelliJ needs to remove the parameters that are part of the data clump and add a new method parameter that replaces the method parameters of the data clump. 
    \item If a variable is used it can be replaced by a getter or setter call. For instance, if the variable \textit{var} is read, and the name of the extracted class is \textit{Object}, any reading of the variable can be replaced by  \textit{object.getVar()}, where \textit{object} is a variable of type \textit{Object} and \textit{getVar} is the getter of \textit{var}. Similarly, an assignment can be replaced by the setter.
    \item If a method is used, several sub steps are needed.
    \begin{enumerate}
        \item First, for each argument provided in a method call it is determined whether the argument is connected to a data clump variable (i.~e. it provides a value to a parameter that is part of a data clump) 
        \item The position of those arguments is stored, and a reference to the argument is stored for further processing.
        \item Since the extracted class is known and existing, a matching constructor is determined to support all arguments to the data clump variables of the method call. 
        \item For each argument to a data clump variable, the argument is inserted into the matching position of the constructor, and the argument is removed in the original method call. 
        \item the constructor call is added at the position of the method  where the parameter of the extracted class is expected. 
        
    \end{enumerate}
    
\end{enumerate}

This approach requires some coordination. For instance, the order of operation is important. Method and field declarations must be updated first because they are needed to perform the refactoring of variable usages and method calls successfully. 

Another aspect where the order of the operation matters is the hierarchy in the abstract syntax tree. Consider the assignment \textit{x=x+1}. In an abstract syntax tree, the reading of the value \textit{x} and the addition of 1 are executed first. It is also at a deeper level of the tree than the assignment. If the assignment were to be replaced by a setter, the syntax tree of the reading expression can be out of sync because it is not linked to the original assignment anymore. Therefore, it is vital to refactor parts of a code with higher depth in the syntax tree before parts with lower depths. 

\section{Integration of Large Language Models}\label{sec:implementation_tools}
Integrating large language models for data clumps refactoring poses the challenge of transmitting source code to the model, and afterwards integrating the changes and  feedback back to the source code. Difficulties arise in determining how to submit the source code to the \ac{LLM} and  the model should reply so that such integration runs as seamlessly as possible. 

The challenges vary depending on the specific uses of the \ac{LLM}. The following applications of models are discussed in this section:
\begin{enumerate}
    \item Detecting data clumps
    \item Choosing the best data clumps (filtering)
    \item Refactoring data clumps
\end{enumerate}

Each usage can be performed independently from each other, but internally all previous steps must be conducted. For instance, a model instructed to refactor data clumps must find them first, then decide which data clumps are worthy to refactor (these could be all data clumps), and then refactor them. Similarly, a model instructed to filter data clumps must find them first, but does not need to refactor them.

Section \ref{sec:input_format} outlines the strategy on how the existing context obtained by previous steps (e.~g. detected data clumps) can be conveyed to the model.

Afterwards, section \ref{sec:output_processing} describes the reverse direction on how the output of the model can be interpreted so that they can be applied to the context or directly to the source code. 

Lastly, section \ref{sec:proposal_handling} explains how multiple proposals by the model can be handled so that the inherent randomness of such models can be utilized. 

\subsection{Input processing}\label{sec:input_format}

\begin{table}[ht!]
    \centering
    \begin{tabular} {m{4cm} | m{4cm} | m{4cm}}
        Detecting & Filtering & Refactoring  \\\hline
         \begin{itemize}
             \item AST
             \item Full source code 
             \item Snippets of methods with at least three parameters, classes with at least three fields
         \end{itemize} & \begin{itemize}
             \item Data Clump Type Context
             \item Full source code
             \item Snippets of source code of the data clumps
         \end{itemize} & \begin{itemize}
             \item Full source code 
             \item Snippets of source code of the data clumps
         \end{itemize}
    \end{tabular}
    \caption{Input categories for different scenarios regarding using \ac{LLM} in data clump refactoring}
    \label{tab:data_clump_llm_input}
\end{table}

The input provided to a model is strongly dependent on the purpose of the model. Table \ref{tab:data_clump_llm_input} shows possible input types for each \ac{LLM} usage. Each input type is discussed below:

\subsubsection{AST}

The \ac{AST} is an effective method to give the model all relevant information about the source code.  The \ac{AST} only contains a reduced structure of the source code and will therefore help to reduce processing cost. However, the \ac{AST} is usually not directly available but must be produced by other tools or ChatGPT, necessitating that source code still be sent to the \ac{LLM}. 

If ChatGPT is also tasked with refactoring, using \ac{AST} may not be as beneficial. Refactoring often involves updating method calls or variable usages, which may not be fully represented in the \ac{AST}. One could hypothetically submit both the \ac{AST} and the source code, enabling ChatGPT to use the \ac{AST} for detection and the source code for refactoring. However, this approach increases costs, and it is unclear whether it could degrade the quality since ChatGPT would need to process more data and establish a correlation between the source code and the \ac{AST}.



\subsubsection{Code snippets}\label{sec:code_snippets}


Another possible approach is to only provides those lines of code that relate to data clumps. As a result, only parts of the source code are transmitted which can save many tokens and helps the model to focus on the relevant parts. These lines might be called \textbf{locations of interest}



The major issue with this strategy is to choose which lines of code to transmit. Only if the data clumps are already known, all relevant locations can be found and transmitted to the model. This does not work if it is the purpose of the model to also to detect data clumps. In this case, one strategy could be to define the locations of interest as lines where either a method with at least three parameters are declared, or a field in a class with at least three fields are declared. This ensures that only potentially relevant lines are considered. If usage information is available, these lines can be transmitted to help refactoring although this is scarcely helpful for filtering and detection purposes. 

Consequently, the idea of the context as discussed in section \ref{sec:context} can be used. Using the most-recent context available, the locations of interest can be determined by taking into account all the information present in that context. For instance, a data clump type context contain information about the detected data clumps and their file path and location in the file. Similarly, the reference context contains the file path and line number of all references of the detected data clumps.



To further improve the quality of \acp{LLM}, the number of lines can be increased. Instead of transmitting only the line of the data clump items,  a small neighborhood or margin around these lines is transmitted, too. This helps the model to gain a better overview about the source code. For instance, these additional lines could include documentation or the usage of variables, thereby helping the \ac{LLM} in its task.

\begin{figure}
    \centering
    \includesvg[width=0.4\columnwidth]{figures/chapter4/margin_effect.drawio.svg}
    \caption{The effect of varying the margin on the available information}
    \label{fig:margin_effect}
\end{figure}

Figure \ref{fig:margin_effect} illustrates the advantage of a larger margin size. The source code shows counters used for testing purposes and a possible way to reset them. The first four fields are part of a data clump (the second class is not shown for brevity). Assume for the sake of this example that only the data clump item \enquote*{afterAll} is transmitted. If the margin is zero, only line 4 would be transmitted to the model (black rectangle). This means that the \ac{LLM} has only a scarce overview over the purpose of the variable.
The green rectangle covers the area if a margin of one is used. Now, line 3 and 5 are also included. Since line 5 mentions something about tests, the model can infer where the data clump items are used and might improve its refactoring. The red rectangle represents a margin size of three. In this case, the full code block is transmitted. Now, the model can observe that it might be useful to include a reset method in the extracted class. 

However, not only data clumps items could be used as the starting point of the additional lines. Many source code files contain a header and import statements in the beginning. This information can be beneficial especially in case of refactoring as the model can better understand the types of the variables or the context of the source code.

In this master's thesis a constant margin is considered. However, variable margins that depends on the programming language or statistics of the source code (average length of a method or documentation) could be considered too.


\subsubsection{Full source code}

An expensive method that works with all three applications of \acp{LLM} is providing the full source code. This means that the source code is transmitted to the model without any modification. While the overhead is large, this method gives the model the largest amount of information so that it might execute the detection, filtering, or refactoring step more reasonably. However, due to context size limitation and resource allocation, providing large files can have a detrimental  effect on the quality. 

Here also, the selection of the files is crucial to ensure sufficient quality. The locations of interest strategy discussed in section \ref{sec:code_snippets} can be used too. However, all line number information is ignored which means that a file is sent to the model even if only one line in that file is a location of interest. 




\subsection{Output processing}\label{sec:output_processing}
An \ac{LLM} on its own cannot change code or control a system since, for the purpose of this master's thesis, it  can only output textual information. Therefore, each output from the model must be received and interpreted  in order to serialize them permanently. Two approaches are discussed. 

Commonly, a human in the loop between the \ac{LLM} and the source code who reviews the output, interprets it and uses the gained knowledge to simplify the workflow. In the case of refactoring, the model can propose changes to the source code but cannot change the source code directly. Instead, a human in the loop has to read the refactoring suggestion, determine whether they are reasonable, and perform the suggested refactoring. In the last step, this also involves finding the affected files, change them, and test the changes. Hence, while an \ac{LLM} can help to make decisions, manual work must still be performed. The benefit of this manual work is that the human in the loop bears the responsibility by using the output from the model which can be one part to ensure the reliability of the refactoring. 

One approach to facilitate these suggestions is to create a mirror file. This means that the original file is untouched, but the changes are written to a similar named but new file. A human in the loop must then manually apply the refactoring suggested in the mirror file to the original file. While this is a more time-consuming task, it allows for more control by the developer. Additionally, \acp{LLM} tend to omit unchanged code. Directly overwriting files will therefore result in much smaller files that have lost much of their contents. Moreover, creating multiple mirror file would be possible allowing the developer to have multiple refactoring option suggested by one or multiple models with various parameters so that there is more flexibility. 
Markdown is one common way to integrate code and text description to explain the human in the loop to decide what steps to perform in order to successfully refactor the data clump.  

\bigskip

With no human in the loop, the output of the \ac{LLM} must be parsed automatically and be used by the respective handler to perform the current pipeline step. Therefore, defining a suitable output  format  is essential as the handler has to rely on the output and cannot adequately deal with an output that does not conform to the specified format. 
\subsubsection{Detection}
If the \ac{LLM} is asked to detect data clumps, the format described in appendix \ref{app:data_clump_format} can be used to ease compatibility to other tools. This form can be clearly defined as it can be easily parsed. A disadvantage of this format is that it can become verbose and contain redundant information. For instance, if four methods  constitute a data clump, there will be six entries in the output format (i.~e. method a-b, a-c, a-d, b-c, b-d, c-d). All these entries must, according to the format, contain the class name, the file path, a unique id, the names and types of the parameters etc. This redundancy is superb for manual refactoring tools as this information can be easily accessed. For instance, a refactoring tool can change the signature of one method independently of another because all relevant information is located within the data clump information for that particular method. 

For an \ac{LLM}, this redundancy can be an obstacle as models have limitation regarding the number of tokens they can output. Every duplication can therefore hinder the model from reporting more data clumps.



As an alternative, to this format, a more relaxed format can be used. Instead of restricting the model to a predefined but lengthy format, the model is instructed to report in a simpler format or no particular format at all. The output can still be parsed to a reasonable degree because it can contain all relevant information for defining the concrete parts of the source code affected by a data clump. For instance, the model might report a file path, a name of the method, and variable names. This information can be combined with the \ac{AST} to reasonably narrow the data clump the model is referring to. From here, the data clump type context can be constructed. While this approach is not absolutely reliable, it offers a good approach to save tokens while finding as most data clumps as possible. However, this only works if the \ac{AST} is available. 

\subsubsection{Filtering}\label{sec:output_format_filtering}
 The output format for reporting results is a \ac{JSON}  consisting of a data clump key, a reason, and a justification. The key depends on the concrete input format the model is provided. For instance, if the data clump type context format is provided, the key is part of the submitted data and the model can reuse it. If pieces of the source code is provided, the key can be easily integrated too. In the full source code case, however, it is more difficult to provide a key, as this key must somehow be placed into the source code (e.~g. as a comment). While the \ac{LLM} can generate its own key for the data clump, it must be mapped back to the original data clump type context so that further steps in the pipeline can continue. One way to solve this issue is to define the key as a portion of the source code where the data clump is located. 

 The justification is an explanation by the model about why it has chosen exactly this data clump. It is directed to a human in the loop, but simultaneously eased identifying a data clump. In many cases, the justification contains the name of the method, or the names of the fields, or the name of a class. This information can be used to determine the correct data clump the model is referring to in case the key cannot be mapped to the  data clump type context. This can be achieved by a simple rating system where the data clump, whose properties (e.~g. class, file name, method names etc.) occurs the most in the justification, is probably the chosen data clump. 

 The reason in contrast is just a single word representing the metrics discussed in section \ref{sec:data_clump_filtering}. For instance, one reason can be \textit{size} or \textit{occurrence}. These reasons can be statistically evaluated or can be used by a human in the loop to decide which data clump to choose. 

\subsubsection{Refactoring}
For refactoring, parsing the output correctly is even more important as it directly causes changes in the source code.

One approach is to use markdown. The markdown format can be analyzed automatically by clearly splitting code and text sections, however it is difficult to apply the code section to the correct location. 

As an alternative, the \ac{JSON} format allows for easier automatic processing of the response as \ac{JSON} data can be parsed without major obstacles and can contain all relevant information. For instance, \ac{JSON} data that contains the path to a file as the key, and the file content as its value can be parsed and the file content can be written to the given location. Some \acs{LLM} also allow to force the \ac{JSON} format so that the response is always valid \ac{JSON}, although it might not be in the requested format. 

The disadvantage of \ac{JSON} is that it is harder to read by the human in the loop. Because \ac{JSON} strings do not support new lines and code files contain many new lines, they are hard to read and understand.
One challenging aspect of this method is that the \ac{LLM} will likely make errors. For instance, the modified files are not compilable as the file might not be complete or the  overwritten files contain explanatory text. A human in the loop must then apply the necessary corrections on the overwritten file.



Therefore, the \ac{JSON} approach should be modified so that not full file contents are returned but replacement instructions. These are instruction given by the model to replace specific lines by a given text, so that only specific lines of a file are modified.  As a result, code lines not related to data clumps can be left untouched minimizing the risk for mistakes. 


Figure \ref{fig:json_based_changes} displays a \ac{UML} sequence diagram that shows how the diff-based approach works on a simple example. Here, the fields \textit{x}, \textit{y}, and \textit{z} are part of a data clump and need to be refactored. 

The handler sends the file or a code snippets to an \ac{LLM}. The model processes the source code based on the instruction and generates a diff instruction. This diff instruction contains a line number, an old content, and a new content. The \ac{UML} note in the bottom right of the diagram shows a simplified diff instruction. Here, it instructs the handler to replace the content \enquote{int x; int y; int z} (which is abbreviated for brevity) in line 1 by \enquote{Point pt}. 

This diff instruction is processed by the handler which changes the source code  and writes the modifications to the file system. 
\begin{figure}[ht!]
    \centering
    \includegraphics[width=1\columnwidth]{figures/chapter4/sequence_diagram_piece_output.drawio.pdf}
    \caption{\ac{UML} sequence diagram of applying a diff instruction (UML 2.5)}
    \label{fig:json_based_changes}
\end{figure}





One major issue is how to correctly apply the suggested changes if the model makes mistakes. 

Two approaches are possible.
In the first approach, the content at lines \textit{}{fromLine} to \textit{toLine} is replaced by \enquote{newContent}. This can be challenging if \enquote{newContent} has more lines than \enquote{oldContent}. Additionally, even if these  changes are correctly applied, later changes in the same file can become more problematic as the line number information might be outdated. 

Alternatively, a replacement method can be used. This means that the string \textit{oldContent} is replaced by \textit{newContent} without considering the line number information. This method prevents the issues of the first approach. Additionally, it might refactor more lines that the \ac{LLM} did not refactor for some reasons. This can have advantages but also challenges. For instance, an instruction to replace all curly brackets (\enquote*{\{}) by an empty string in a Java file can lead to non-compilable programs.  
Additionally, this does not work if \textit{oldContent} for some reason does not exist in the file. For instance, the model might have added extra whitespaces to the old content so that a simple search-and-replace-strategy does not work. Furthermore, if \textit{oldContent} is an empty string, replacing can lead to file corruption as the replacement operations are not developed for such an scenario. 

On the other hand, if multiple methods share the same parameters, the replace approach can change them all at once even if the output of the model does not suggest such changes. While adhering to the output of the \ac{LLM} is useful, it is restricted due to its output size limitations. Consequently,  this method can forestall incomplete refactoring. 

As a result, the decision which strategy to use is complex and can strongly impact the number of changes until the source code can build again. In general, the shorter the old content is, the more likely should the first approach be used, but exact criteria cannot be given. 


In this master's thesis both approaches are employed. The second approach is always tested first if the changes exceed a threshold of ten characters. If this approach does not work (e.~g. \enquote{oldContent} does not exist in the file, the line-by-line replacement method is used.  

\subsubsection{Handling invalid \ac{JSON} objects}

When using the \ac{JSON} output format, it is essential to deal with edge cases where the output is not valid \ac{JSON}. This problem can happen in two scenarios.

Firstly, not all \acp{LLM} support \ac{JSON}-only modes. This means that that the content of the model cannot be forced to contain only \ac{JSON} but might contain additional unparsable text that cannot be parsed. 

The model used in this thesis supports the \ac{JSON}-only mode. Nevertheless, it is still useful to keep these edge cases in mind. This edge case can be circumvented by finding the first opening curly brace \enquote{\{} in the response and the last closing curly brace \enquote{\}} because the string between should be valid \ac{JSON}. 
As a result, any text before or after the \ac{JSON} is ignored and cut off. 

Secondly, one problem for \acp{LLM} is that they generate token by token and therefore cannot pre-plan their output. Usually, this does not cause problems. 

However, as outlined in section \ref{sec:llm_challenges}, the maximum output size of a model is restricted so that a model cannot generate large output at once. The model can be prompted to continue the output, but this requires additional prompts. Additionally, these interim prompts must be combined into one so that they can parsed.

If the output format is \ac{JSON}, handling interruptions can become problematic because \ac{JSON} syntax requires a precise structure. For example, if the closing curly braces are missing, standard \ac{JSON} parsers will struggle to process the data and will typically generate an error. This strict syntactic requirement means that even minor deviations can disrupt \ac{JSON} data handling.

The easiest solution to this problem is to simply ignore the output as the response was invalid. 
However,  one can still try to parse the output as the incomplete \ac{JSON} might still contain relevant data (e.~g. refactoring instructions). While the refactoring is probably incomplete, it might be a good start and possible errors can be fixed in the validation step.


\subsection{Proposal handling}\label{sec:proposal_handling}

One issue that might occur while working with \ac{LLM} is that their output quality can vary. The same prompt can generate various outputs of which some might be useful and some are not useful. This, however, is also an advantage of \acs{LLM}. Refactoring data clumps can be performed in many different ways and therefore, the first output might not be the best output. 

Hence, a proposal system is useful where the model is asked multiple times for a solution to a query (e.~g. data clump refactoring). These proposals can be handled parallel or serial.  

\subsubsection{Parallel Proposals}
The model generates multiple independent proposals. This means that in each proposal the model has no memory of its previous proposals. At the end, the best proposal is chosen and applied to the source code.

The issue is to determine the \enquote{best} proposal. One approach could be to determine the proposal that causes the least compiler errors. While this idea might be helpful to reduce the manual work to correct errors, it also can tend to produce proposals that change as few code as possible. For instance, refactoring nothing can cause no compiler errors.

As an alternative, one could count the number of changed lines in a proposal and rate proposals with more changes higher than those with fewer changes. This idea rewards proposals that refactor all locations of interests more than proposals that only refactor only a small part of the program while the data clump exists in other known parts too. 


Another possible idea is to let the user decide. For instance, for each proposal, a new branch in the \ac{VCS} could be created so that a user can choose manually which version would be best one. After choosing a particular branch, the changes could be merged back to the original branch. Alternatively, an interactive dialog  could allow the user to zap through all proposals without changing branches. When selecting a proposal, the changes from the previous proposal are deleted and the new proposal is applied to the source code. However, this strategy requires active human intervention. 

This master's thesis does not attempt to explore which proposal selection strategy is suitable or whether other strategies would be better. 

\subsubsection{Sequential proposals}

With sequential proposals, one conversation is hold and the context is kept so that that the \ac{LLM} does not forget its previous output. Using this idea, not the first proposal is chosen but the last one after the \ac{LLM} has been given enough chances to correct errors.

Handling sequential proposal can be a combination of refactoring and validation handlers. At first, the usual refactoring process is performed. This could be a manual refactoring (e.~g. PSI) or a refactoring via an \ac{LLM}. Then, the refactored code is built. If there a building issues (e.~g. compiler errors or failing tests), the model is instructed to fix them. From now on, only the \ac{LLM} performs the refactoring even if a manual approach was used. Then, the code is built again, and the process starts again if there are errors. 



It should be noted that at some point a proposal should finally be chosen as the \ac{LLM} might not always find a solution that compiles even if infinitive steps were available. Similar to a an optimization algorithm, it can be struck into a local optimum and cannot suggest reasonable changes so that the program compiles again.  
\section{Fltering approaches for files and data clumps}
\label{sec:data_clump_filtering}

As noted in section \ref{sec:pipeline_steps}, filtering can be used to reduce the data size of the refactoring process and only refactor those data clumps that are worthy to refactor. It is important to note that the filtering process happens at two stages.

In the first stage, no data clumps have been detected yet, and only file paths are known. Therefore, only a limited set of information is available. 

In the second phase, data clumps have already  been detected given the (possibly shrunk) set of files. Now, information about the data clumps can be used  to decide which data clump will be refactored in later phases.

In both phases, the same type of filtering objects are used because  a separate design for file filtering objects and data clump filtering objects would be over-engineering. A file filtering object could work with data clumps becaus ethe have an associated file path, however because a data clump could have two different file paths, the filtering object must take care of that issue. Conversely, a data clump filtering object might work for files only in some circumstances where the file path is sufficient. 

Hence, each filtering objects must indicate whether it supports file paths, data clumps or both, so that the filtering objects can be correctly used. 

In this section, file paths and data clumps are generalized as \textbf{items}.

Two concepts to  reduce the number of items can be distinguished.
\subsection{Ranking} \label{sec:metrics}
In the ranking approach, each data clump is scored using a metric. A higher score is better than a lower score.  Then, the data clumps are sorted based upon their score in a descending order. Finally, the first  \textit{n} data clumps are retained while the remaining data clumps are filtered out. With this ranking approach only important data clumps are retained. The importance of a data clump is determined by a metric, and the \enquote{survival} of a data clump depends on how many other data clumps with better scores exists.

In the following, the implemented metrics are discussed:

\subsubsection{Size of a data clump}

The size of a data clump means the number of variables associated with it. For instance, a data clump with the variables \textit{x}, \textit{y}, and \textit{z} has the size 3. 

Prioritizing data clumps with large sizes can be beneficial because removing them reduces the code size more significantly which helps to improve readability. Additionally, they are more likely to be perceived as a data clump because it is easier to notice two methods that share more than three parameters in contrast to methods sharing only three parameters. 

On the other hand,  it can become more difficult to find a suitable name for the extracted class the more parameters are shared because the probability of a common topic can be lower. 

\subsubsection{ Data clump occurrence}

The occurrence metric describes how often a type name key of a data clump appears throughout the software project to be analyzed. A high value is a strong indicator that refactoring is necessary because it demonstrates how strongly the code size can be reduced. Additionally, this metric is hard to measure manually because no developer will have an overview on the whole project and can count this occurrence making it more difficult to realize that a data clump has a high occurrence.  

On the other hand, refactoring these data clumps require that more lines of the project must be changes which needs to be verified and tested. 

\subsubsection{Affected Files}

The number of files affected by a data clump is somewhat  similar to the occurrence metric. However, in this metric each file is counted exactly once. If a data clump occurs in one file ten times and in another file 5 times, the occurrence metric would count all occurrences while the affected files metric would return two as only those both files are affected. 

Especially, if complete files are transmitted to an \ac{LLM}, this metric should be inverted meaning that low values should be ranked higher than higher scores. An \ac{LLM} only has a limited context window and the more files are transmitted the higher are the chances for errors.  

\subsubsection{Update time}

The time or frequency of when a particular file is updated can also be a metric to ascertain whether a refactoring is warranted. These information can be obtained from \ac{VCS} like Git as they are part of commits.

Files that have been recently updated can have new data clumps that should be refactored. These data clumps can be refactored more easily because it is more improbable that they have manifested so that other parts of the project or outside project depend on the existence of the data clump. 

On the other hand, files that were untouched for longer times might also contain data clumps that exist but are not refactored. One reason can be that developers are not familiar with these source files and are afraid to change them. Depending on the concrete type of data clumps, they can nevertheless be refactored by automatic tools so that developers do not need to read this unknown source code.


\subsubsection{Combining metrics}

One special metric is dependent on other metrics because it only creates a weighted sum of other metrics therefore allowing multiple metrics to be combined. 

\subsection{Filtering}
In the filtering approach, a binary condition is tested on each data clump individually. if the data clump does not meet the criterion, it is removed from further consideration. hence, each data clump is analyzed independently. Theoretically, this could mean that all data clumps are filtered out if the filtering conditions are too lax and apply to all data clumps. 

These filters can be combined by logical operation (and, or), so that more complex filtering rules are possible. 

The following subsections presents some  criteria for filtering data clumps
\subsubsection{Ignore abstract methods}

Abstract methods or non-implemented methods in interfaces only describe a contract without functionality which must be implemented by derived classes. Therefore, the signature of such operations is essential because it describes the contract which must be obliged to by derived classes.

Hence, changing the signature of abstract methods is often avoided as derived classes or classes implementing the interface must be modified too. These classes may reside outside the scope of the abstract class or interface. Therefore, refactoring the signature of abstract methods has a higher chance of errors because the derived classes may not be known or cannot be changed. 

Hence, it can be useful to exclude abstract methods for purpose of data clump refactoring to prevent these issues. 


\subsection{Combining metrics and filters}

The differentiation between filters and metrics can be relaxed by using metrics as a filters, and filters as metrics thereby enabling the construction complex filtering methods. 

In the first case, filters employ the metrics described in section \ref{sec:metrics}, a comparison operator, and a threshold value. If the metric applied to a data clump compared with the threshold value using the comparison function returns true, the data clump remains, otherwise it is filtered out. For instance, the metric can be the data clump size, comparison function can be the greater-than function, and the threshold value can be three. In this example only data clumps with more than three variables will remain for further consideration. 

Vice versa, a filter can be used as a metric. If a filter returns true on a specific item, the metric returns a different value than it would if the filter returns false. As result, the binary classification by a filter is still employed, but items that do not fulfill the criteria of the filter still has a chance not be filtered out. 


\section{Configuration and usage of the main program}\label{sec:config}
This section outlines how the main program can be configured and used.

\subsubsection{Configuration}
An important aspect for the usability of the tool is the possibility to configure the tool for the user's need. 

Since the goal of the tool is to allow the combination of multiple services and other tools in order to find and refactor data clumps, the user must be able to define which handler deals with which step  as outlined in section \ref{sec:pipeline}.

The configuration is provided by a \ac{JSON} file whose location needs to be provided to the tool via a command line argument. It might be argued that providing the configuration directly via the command line is better suited than a separate configuration file because they do not require the creations of files and are easier for users who start the tool just once.

However, configuration files are persistent and especially \ac{JSON} can be more easily structured so that they are easier to understand. As a result, only \ac{JSON} files will be used  for the configuration. 


Listing \ref{lst:config} shows an example configuration file:
  \begin{figure} [htbp!]
			\lstinputlisting
			[caption={ Example configuration file},
			label={lst:config},
			captionpos=b, basicstyle=\footnotesize, tabsize=2, showstringspaces=false,  numbers=left]
			{figures/chapter4/config.json}
		\end{figure}


In the beginning, the programming language is defined (l.~1). 

Then, the user can define objects. An object can be used by a handler to perform sub-steps, but declaring an object in \ac{JSON} does not necessarily mean that it is used. 

Here, one object is declared (l.4). The \textit{AbstractLanguageModel} represents the interface to the \ac{LLM}. In this case, it is an instance of the type \enquote{ChatGPTInterface} (l.5). The object is given arguments via the \enquote{args} attribute (l. 6). In this case, model name and temperatures are given (l. 7-8).  This instance is stored internally and can be queried for.   

Whenever a handler needs the services of another object, it has two possibilities.
\begin{enumerate}
    \item The handler knows the exact concrete type of the object it needs. For instance, handlers that filter files or data clumps can be given the names of filters and metrics that they should use. The handler can receive the respective object by querying the known name. 

    \item The handler does not know the concrete type but requests access to a service. In this example, a handler does not need to know that ChatGPT is used as the \ac{LLM}. It can request the object by querying for the abstract class name \enquote{AbstractLanguageModel}. This represents the classical dependency injection approach as the handler does not need to care about the specific model, but only that the \ac{API} contract is respected. The disadvantage of this design strategy is that only one  concrete type for each abstract type can be defined. If someone wants to use an \ac{LLM} for detection, and another model for refactoring, the current design would not allow this. 
\end{enumerate}




Then, the pipeline is defined (ll. 11-15). It consists of a single handler in this example. The keys of the pipeline are the names of the individual steps to which a handler object is assigned. In this case, the step is the data clump detection step (l.~12) and the  handler uses an \ac{LLM} for detection (l.~13). The concrete \ac{LLM} does not need to be provided as the handler can query the object. 


It should be noted that the order of the handlers in the configuration  does not matter because the execution order is constant and, in most cases, each step requires the context of a previous step so that parallel execution or vice-versa execution of steps is not possible. Only in the case of usage finding and name finding would a parallel execution make sense because none of these steps is dependent on the other.  However, this exception is not implemented.


\subsubsection{Running the main program}


If a valid configuration file has been created, the main program can be started via \textit{npm run run <config> <project-location>} where \textit{config} is the location of the configuration file and \textit{project-location} is the path to the project to be analyzed. The order of these arguments can be swapped. 

With these arguments given, the following steps are executed:

\begin{enumerate}
    \item The main program parses the configuration file and initializes the object with the given arguments. It also assigns the handlers to the respective steps.
    \item  The validity of the pipeline is checked. For instance, it is tested whether all required programs are installed.
    \item An initial code obtaining context is generated.
    \item The pipeline is executed. For each executed pipeline step:
    \begin{enumerate}
        \item Checks whether there is a serialized context for this step and whether the handler accepts that this context is sufficient. If these criteria are met, the handler can be skipped.
        \item If none of these conditions are met, the handler is executed.
        \item the context returned by the handler is used as the new context.
        \item If supported by the respective context, it is serialized.
    \end{enumerate}
    \item The main program exits. The running time of each step is serialized for evaluation purposes. 
\end{enumerate}

It can be seen that due to the modular design, the algorithm for the main program is fairly small. However, it performs a critical part of the refactoring process as it uses the responses from each handler to decide how to continue with the execution of the program. 
