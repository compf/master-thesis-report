\chapter{Implementation}\label{chapter:implementation}

%In this chapter, the the design choices and considerations discussed in chapter \ref{chapter:conception} will be implemented. Firstly, In section \ref{sec:step_impl} an overview will be given over the implementation of handlers that do not  strongly rely on \acp{LLM}. After that, strategies to filter data clumps are presented in section \ref{sec:data_clump_filtering}.   Then, in section \ref{sec:config}, the configuration of the tool and how to use it is explained. 

In this chapter the design choices and considerations discussed in chapter \ref{chapter:conception} are expanded to develop a fully functional automatic tool. 

In the beginning, section  \ref{sec:step_impl} discusses how the traditional tools are integrated into the pipeline so that they can participate in refactoring data clumps. 

Afterwards, section \ref{sec:implementation_tools} outlines how a \ac{LLM} can work with source code. To facilitate this, the model must somehow gain access to some representation of the source code. Then, the reverse direction is discussed as the output of the model must somehow be integrated into the context, or directly as refactoring into the source code. 

Because of the limitations of \acp{LLM}, filtering is an important part of the pipeline. Hence, the architecture of the filtering process and examples of various filters are explained in section \ref{sec:data_clump_filtering}.

Lastly, section \ref{sec:config} deals with the configuration of the tool. This includes how a step is linked to a handler via a configuration file and  how other objects like the \ac{LLM} to use are set up. Afterwards, there is  a brief overview of how the tool can be started and what steps happen in the background.


\section{Implementation of handlers} \label{sec:step_impl}
In this section, the implementation of the handlers is discussed. These handlers do not or only scarcely use the features by \ac{LLM}. 
\subsection{Data Clump Doctor}

The main tool to detect  data clumps is the \textit{DataClumpDoctor}. As this program is written using the NodeJS platform it can be directly integrated into the main program blurring distinction between handler and service.

The \textit{DataClumpDoctor} performs both the \ac{AST} generation step and the data clump detection step although these two steps are internally separated.

If a \ac{AST} context is already present, it re-used. Otherwise the \ac{AST} context is generated. Here, the file filtering inclusion and exclusion rules must be applied as only the generated \ac{AST} files form the foundation for the \textit{DataClumpDoctor} to detect data clumps. Files  not represented by an \ac{AST} file are ignored. Since the detector internally uses \textit{PMD}, inclusion and exclusion rules can be set by creating a \enquote{custom-java-ruleset.xml} file. This file is located inside the Java-Archive file used by the detector, and must be updated whenever \ac{AST} files are generated to apply the current filtering rules. 

After that, the \textit{DataClumpDoctor} loads the \ac{AST} files and detects the data clumps by comparing the identifiers and types of variables that might constitute a data clump, thereby requiring strict equality. These detected data clumps are saved in a separate file and can be re-used in the pipeline. 

One issue with the \textit{DataClumpDoctor} is its handling of large projects. While a concrete \ac{LOC} number cannot be given, at some points analyzing large projects becomes infeasible leading to crashes or prolonged running times. This can be explained by considering that  the number of detected data clumps grows quadratically with the actual number of data clump holders. Due to the structure of the reporting format as discussed in appendix \ref{sec:data_clump_format}, there is a significant amount of repetition. For sufficiently large projects, this can be a problem it the data clump information needs to be serialized as NodeJS is not optimized for these tasks. While changing the format is one solution to this problem, it can be argued that using previous file filters is more suitable as the data clump detection results from large projects are difficult to evaluate. 

\subsection{Name finding}
Finding a suitable name for the soon-to-be-extracted class is the next step  that must be implemented. Two approaches are chosen.

The first approach is the classic approach that has been used before the advent of \acs{LLM}. By concatenating all field names of the extracted class, a valid class name can be generated. For instance, a class with the fields \textit{x}, \textit{y}, m\textit{z} could be named \textit{XYZ}. In this example, the field names are capitalized and directly joined without any separator, however, other options might be better dependent on the project's style guidelines. 

The advantage of this manual method is that the generated names are very unique and with a high probability will not conflict with other names because they are very artificial. Nevertheless, they lack creativity and in most cases, a developer will need to choose a better name in order to improve readability. 


On the other hand, a suitable name for the extracted class of a data clump can be chosen by using the creativity of \acs{LLM}.  For this the model has to know the names of the variables of the data clump. It is also useful to include the qualified type of each data clump item because this type contains additional information to generate a more suitable name. For instance, the qualified name could contain the name of the project to analyze, the location of the type,     and the range of possible values that the variable can have.

While using \acs{LLM} for name finding, excessive name caching can be useful to save costs and resources. The same combination of data clump items will probably result in the same class name. In order to prevent the extraction of classes with similar purposes, it is therefore useful to only ask the model if the types-names identifier has not occurred before. 

A related issue to name finding is the location of the extracted class. Here, one has to differentiate  between data clumps existing in a single file and data clumps connecting multiple files. In the former case, the location of the extracted class can be an already existing file, namely the file of the data clump  while in the latter case, a new class should be created.

Using an existing class can be problematic because the succeeding step (class extraction) must be mindful that an existing class is used as it could override the file. Additionally, integrating the new class into the existing class can be challenging. Should the new class be an inner class of some class? In Java, should the inner class be static?  Creating an inner class require strategic syntactical modification of the source code which would require specialized handlers. As a result, they are not implemented in this master thesis, but could be. 

On the other hand, using a completely new file is easier to implement. However, one has to choose the path of the extracted class. For instance, a data clump might be spread over \textit{n} files each of which is in a separate directory, thereby creating a theoretical possibility of \textit{n} candidates as the output directory. 
One has also be mindful about any name conflicts that might occur. If the suggested name already exists, it will lead to conflicts.


\subsection{Class extraction}

Generating the actual class is a task that can be done manually if the names and types of the fields are known and the name of the class is known. However, the specific syntax depends on the programming language so that each programming language must have its own class extractor.

Other issues should also be taken into consideration. For instance, whether the class body at first contains the declaration of fields, then the getters, setters, and then the constructor, or another order is better depends on the project's style. For instance, the constructor could be declared directly after the fields.

Additionally, not every part of the extracted class might be mandatory. For instance, some fields will never be modified so that a setter would be unnecessary. Furthermore, the extracted class should conform to code styles guidelines. 

Two approaches are implemented. In the manual method, the class is extracted using simple text manipulation. The order of the members of the class can be configured.  However it would be challenging to determine whether a setter for a field is needed. Hence, in the manual approach all getters and setters are generated.

Here also, the creativity of an \ac{LLM} can be useful. Given a suitable context, an \ac{LLM} can return the source code of the extracted class without being restricted to a template as the manual approach is. For instance, instead of generating a class, the model could create a Java record which is a simplified version of a class. 

\subsection{Reference Finding}

Other relevant handlers deal with the finding of references. Only if all references are correctly resolved, the data clump refactoring can proceed smoothly. 

\subsubsection{Via \ac{LSP}}
In order to refactor data clumps, all relevant locations of interests (references) must be identified. If the IntelliJ plugin is used for refactoring, this step can sometimes be left out. However, since this plugin has some issues and the refactoring might be performed by an \ac{LLM}, it is beneficial to implement a service for reference finding. 

The \ac{LSP} is one method to find these references because it works reliable and is also available for other programming languages.

Four kinds of references are relevant for data clump refactoring.

\begin{enumerate}
     \item A  variable (field or method parameter)  is declared
    \item A variable is read from or assigned to
   
    \item A method is declared or overridden
    \item A method is called
\end{enumerate}
To facilitate the use of \ac{LSP}, the server is started and initialized. After that, a request to obtain references for each (filtered) data clump is sent to the server.
Any reply by the \ac{LSP} is intertwined with the associated data clump thereby creating the reference context.

 
\subsubsection{Primitive reference finding}
This handler works with all programming language as it does not parse the source code at all. Instead, it searches for the identifiers that are part of a data clump (e.~g. method name, variable names). 

This simple approach has two issues. First of all, one must determine where to search. Searching the whole project will require a significant amount of time. Alternatively, only the files containing the data clump or the respective folder can be traversed.

Moreover, this method will lead to more false positives because it disregards the scope of variables. For instance, searching for the variable name \textit{foo} will also match local variables named \textit{foo}. Therefore, a match might not be part of a data clump and refactoring this match can lead to more harm than good. 

This handler works best if combined with an \ac{LLM}. The model can decide for each reference whether it is truly relevant for data clump refactoring purposes. It might also propose a more sophisticated refactoring idea by using these non-data-clump references. 

\subsection{Refactoring by IntelliJ} \label{sec:intellij_refactoring}

For  manually refactoring data clumps, the  \ac{API} from IntelliJ mentioned in section \ref{sec:psi} can be used because it allows for easy modification of the source code that does not result in faulty code. 


\begin{comment}The reasons for this are difficult to determine and the documentation is scarce, so the the PSI approach seems to be only suitable for projects created via IntelliJ or correctly initialized by IntelliJ with the required meta data. Gradle and maven projects are therefore not suitable for the full refactoring step. 
\end{comment}


The basic approach for refactoring is based on modifying references. A  reference is an information about the usage of an element. Element in this context could mean a class, a method, or a variable.
In many cases , IntelliJ can find all references automatically and categorize them.

However, this does not always happen. A project can be loaded wrongly so that  finding references of a method or field can lead to undefined behavior. Sometimes all references are correctly found, sometimes only a subset of the references are found, and sometimes no even no references are detected. This could be explained by invalid caches or unsupported projects although it is difficult to determine whether there are reference errors as no ground truth exists.

One solution for these issues is to open the project in IntelliJ at least once  on the same computer running the tool, so that all references can be loaded. Therefore, using IntelliJ as the main \ac{IDE} of the project to analyze will maximize the chance of finding all references



If this approach does not work, the context can be utilized. If a previous created a context that contains reference information, the IntelliJ plugin can use this context to perform the refactoring. This approach represents the pipeline idea that many tools need to work together to achieve a common goal.


Whether such manual search is faster than the combined use of \ac{LSP} and IntelliJ is difficult to determine. On the one hand, can external services be faster because they are implemented better (i.~e. more sophisticated algorithms). On the other hand, starting two services leads to more overhead. 



Regardless of how the reference issue is solved, IntelliJ can now perform the refactoring. Depending of the category of a reference, IntelliJ needs to perform distinct step. 
\begin{enumerate}
    \item In case of a declared field, the field can be deleted because it is part of a field data clump. Now it can be determined whether the class contains already the new field that replaces the fields of the data clump.
    \item If a method is declared, IntelliJ can modify the signature of the method. This can be the original method or an overridden one. IntelliJ needs to remove the parameters that are part of the data clump and add a new method parameter which replaces the method parameters of the data clump. 
    \item If a variable is used it can be replaced by a getter or setter call. For instance if the variable \textit{var} is read, and the name of the extracted class is \textit{Object}, any reading of the variable can be replaced by  \textit{object.getVar()} where \textit{object} is a variable of type \textit{Object} and \textit{getVar} is the getter of \textit{var}. Similarly, an assignment can be replaces by the setter.
    \item If a method is used, several substeps are needed.
    \begin{enumerate}
        \item First, for each argument provided in a method call it is determined whether the argument is connected to a data clump variable (i.~e. it provides a value to a parameter that is part of a data clump) 
        \item The position of those arguments is stored and a reference to the argument is stored for further processing.
        \item Since the extracted class is known and existing, a matching constructor is determined to support all arguments to the data clump variables of the method call. 
        \item For each argument to a data clump variable, the argument is inserted into the matching position of the constructor, and the argument is deleted in the original method call. 
        \item the constructor call is added at the position of the method  where the parameter of the extracted class is expected. 
        
    \end{enumerate}
    
\end{enumerate}

This approach requires some coordination. For instance, the order of operation is important. Method and field declarations must be updated first because they are needed to perform the refactoring of variable usages and method calls successfully. 

Another aspect, where the order of the operation matters, is the hierarchy in the abstract syntax tree. Consider the assignment \textit{x=x+1}. In an abstract syntax tree, the reading of the value \textit{x} and the addition of 1 is executed first. It is also at a deeper level of the tree than the assignment. If the assignment were to be replaced by a setter, the syntax tree of the reading expression can be out of sync because it is not linked to the original assignment anymore. Therefore, it is vital to refactor parts of a code with higher depth in the syntax tree before parts with lower depths. 

\section{Integration of large language services}

Since \ac{LLM} like ChatGPT are a major part of this master thesis, the issue of integrating \ac{LLM} is another part of the concept. The following problems must be addressed
\begin{enumerate}
    \item How should an interface to a \ac{LLM} look like?
    \item How should the conversation with a \ac{LLM} be performed?
    \item How are messages from and to the model structured?
    \item Where should instructions be stored and processed so that they can be sent to a \ac{LLM}
\end{enumerate}
Issue 1 and 2 will be addressed in section \ref{sec:llm_interface}. The issue 2 3 will be discussed in section \ref{sec:llm_msg_structure}. Lastly, issue 4 will be dealt with in section \ref{llm_msg_storage}.
\subsection{An interface for large language models}\label{sec:llm_interface}

Since the market for large language model is constantly expanding in just a few years, designing a interface for communication is challenging. As a result, only the core functionality can be modeled by an interface in order to keep compatibility and ease extendability. 

An interface to a \ac{LLM} should support providing messages. These messages are provided by the user and can be a instruction (see  section \ref{llm_msg_storage}), data or any other relevant information. After a message is stored,  no modifications of the message content can occur so that that necessary text transformation must occur earlier. 

Providing a message does not mean that the message is processed by the \ac{LLM} but is kept until further instruction. Thus, a user can prepare multiple messages before sending them to the model.

If the user decides to send the messages to the mode, another operation can be used. This operation sends the accumulated message to the large language model and waits for the response, so that the operation is synchronous. While an asynchronous approach would also be feasible, in most cases the data clump detection and refactoring process cannot proceed without the relevant information from the model so that waiting is tolerable. 

After sending and receiving the messages, the response from the model can be returned. Now the interface must deal with the messages it has accumulated. Since most models have no memory, the messages must be sent again if they are still relevant for future requests. However, storing and resending messages can cost more so that this should not be done always.

Therefore, the caller of the sending operation has the possibility to clear the previous messages after the \ac{LLM} has responded or keep them, and can therefore decide what to do.

\subsection{A message format for large language models} \label{sec:llm_msg_structure}
The structure of the messages to an \ac{LLM} is another issue to handle. Each model has its own requirements on how a request must be sent to it and how it will respond so that a general message structure must be developed. However there are similarities. Each model differentiates between requests by the user and the responses and represents the messages in a chronological manner, the most recent message is the message with the highest index. 

As a result, a simple message format can be an array of message object. Each message object is either a system message, an input message, or an output message. Input messages are generated by the user while output messages come from the model. System messages can be used to submit general instructions.  Each of these message type has a representation in an \ac{LLM} so the respective handler must perform a conversion. 

A message object may contain multiple messages as it uses a string array. This is useful if multiple messages have a connection and need to be sent at the same time. For instance, if a user wants to transfer the file contents of a project to a \ac{LLM}, he can transmit each file within a a single message object. This not only helps to improve the performance a little bit but allows for easier management of messages since messages are grouped by request. 

\subsection{Resources management}\label{llm_msg_storage}

Another issue that arises while using \ac{LLM} is the management of resources. Resources can either be static or dynamic. 

A static resource can be an instruction or additional context to be transmitted to an \ac{LLM}. A static resource is independent of the project to analyze and must already exists before the data clump refactoring process begins. For instance, examples of data clumps or the instruction to refactor data clumps do not use any information from the project to analyze, but are generally written. 

 Similarly to resources like textures, 3D models, sound data, images etc., they should be separated from the code \cite{separate_code_data}. As a result, separate text files for the instructions are better as they can be distributed and modified more efficiently, especially if other persons or entities create the instruction prompt.

An instruction is often not a single resource but a composition of many resources or other data. For instance, if an instruction contains an code example, it is reasonable to split the code example into a different file to reduce the size of a single file. This  also allows easier modification of the example with an IDE because combining instruction text and source code would lead to compiler errors.

As a result, an instruction resource may need to hold references to other files (e.~g. source code) or references to other data.
For the \ac{LLM}, the instruction should be complete such that it contains the whole instruction with the content of all referenced files and other information.

As a result, two perspectives need to be taken into consideration. From a user perspective, an instruction file should be as modular as possible as explained above. From the perspective of a large language model, an instruction needs to be complete. 

These two perspectives can be reconciled by a template model. The instruction file can be considered as a template. It does not contain the complete instruction that will be sent to a \ac{LLM} but a mixture of actual text and references.

When loading the instruction, all references must be correctly mapped with the correct content so that it can be sent to the model and be correctly interpreted. 

A reference to a raw string is in the format \enquote{\$\{id\}} where \enquote{id} is an identifier. When loading the template file a, string must be provided that replaces this reference.  A reference to a file is in the format \enquote{\%\{id\}}, so it starts with a percentage sign. On loading the template, a path to a file must be  provided and the content of that file replaces the reference. 

Listing \ref{lst:nstruction_template} illustrates an example instruction file.  The instruction prompts the model that code files will be provided (l.~1) and that all data clumps in those source code files need to be detected (l.~4). It also informs the model that examples of data clump will be provided (l. 6 and 11-12) and describes how the response by the \ac{LLM} should be structured (l~7-9). 

However, the examples and output format are not directly specified in the instruction file but are referenced. For instance the text \enquote{\%\{output\_format\}} will not be sent to the model but replaced by the actual output format that is stored somewhere else. The same applies to the examples. Also the specific programming language (e.~g. Java) is not directly defined by the instruction but will added when the instruction is sent to a large language model.

This allows for more flexibility since a single instruction files can be used for multiple programming languages and scenarios. However, it requires more configuration as outlined in section \ref{sec:config}.
\begin{lstlisting}[caption={Instruction file example}, label={lst:nstruction_template}, captionpos=b, numbers=left, ]
I will provide you one or more ${programming_language}
code files.
Find all data clumps in the respective files.

Examples of data clump are provided below.
Use the following JSON format for the output:
## JSON
%{output_format}

## Examples
%{examples}
\end{lstlisting}


\hfill
\subsubsection{Dynamic resources}

In contrast, dynamic resources represent information from the analyzed project. In the end, they are somewhat are representation of the context. However, directly transmitting the context is not useful, as it might contain unnecessary information, or some information must be gathered by combining information from multiple context layers. Therefore, it is the duty of the respective handler to use the existing context and generate messages that contain all relevant information. These messages is appended to the static resources at the end and then transmitted to the model. 

One issue that should be taken care for is that the dynamic resources and the static resources must be compatible for optimal or even usable results. The concept does not guarantee that as it would be unfeasible to implement. For instance, the static resource (instruction) can state that complete files are transmitted to the model, while a handler only submits pieces of the source code or some other representation of the code which can lead to misunderstandings and can decrease the detection and refactoring quality. 

\section{Filtering approaches for files and data clumps}
\label{sec:data_clump_filtering}

As noted in section \ref{sec:pipeline_steps}, filtering can be used to reduce the data size of the refactoring process and only refactor those data clumps that are worthy to refactor. It is important to note that the filtering process happens at two stages.

In the first stage, no data clumps have been detected yet, and only file paths are known. Therefore, only a limited set of information is available. 

In the second phase, data clumps have already  been detected given the (possibly shrunk) set of files. Now, information about the data clumps can be used  to decide which data clump will be refactored in later phases.

In both phases, the same type of filtering objects are used because  a separate design for file filtering objects and data clump filtering objects would be over-engineering. A file filtering object could work with data clumps because a data clump has two associated file paths. Hence, these file paths must be dealth separately and the filter must combine the two (possiblly different) results.

Hence, each filtering objects must indicate whether it supports file paths, data clumps or both, so that the filtering objects can be correctly used. 


Two concepts to  reduce the number of data clumps can be distinguished.
\subsection{Ranking} \label{sec:metrics}
In the ranking approach, each item is scored using a metric. A higher score is better than a lower score.  Then, the data clumps are sorted based upon their score in a descending order. Finally, the first  \textit{n} data clumps are retained while the remaining data clumps are filtered out. With this ranking approach only important data clumps are retained. The importance of an item is determined by a metric, and the \enquote{survival} of an item depends on how many other data clumps with better scores exists.

In the following, the implemented metrics are discussed:

\subsubsection{Size of a data clump}

The size of a data clump means the number of variables associated with it. For instance, a data clump with the variables \textit{x}, \textit{y}, and \textit{z} has the size 3. 

Prioritizing data clumps with large sizes can be beneficial because removing them reduces the code size more significantly which helps to improve readability. Additionally, they are more likely to be perceived as a data clump because it is easier to notice two methods that share more than three parameters in contrast to methods sharing only three parameters. 

On the other hand,  it can become more difficult to find a suitable name for the extracted class the more parameters are shared because the probability of a common topic can be lower. 

\subsubsection{ Data clump occurrence}

The occurrence metric describes how often a type name key of a data clump appears throughout the software project to be analyzed. A high value is a strong indicator that refactoring is necessary because it demonstrates how strongly the code size can be reduced. Additionally, this metric is hard to measure manually because no developer will have an overview on the whole project and can count this occurrence making it more difficult to realize that a data clump has a high occurrence.  

On the other hand, refactoring these data clumps require that more lines of the project must be changes which needs to be verified and tested. 

\subsubsection{Affected Files}

The number of files affected by a data clump is somewhat  similar to the occurrence metric. However, in this metric each file is counted exactly once. If a data clump occurs in one file ten times and in another file 5 times, the occurrence metric would count all occurrences while the affected files metric would return two as only those both files are affected. 

Especially, if complete files are transmitted to an \ac{LLM}, this metric should be inverted meaning that low values should be ranked higher than higher scores. An \ac{LLM} only has a limited context window and the more files are transmitted the higher are the chances for errors.  

In a similar ways, instead of counting the files, the sum of the file size could be determined so that larger file sizes have a greater impact.

\subsubsection{Data clumps per file}

Instead of considering how many files are affected by a data clump, one could also question how many data clumps are in one file. If a file contains relatively many data clumps, it can be refactored by fewer transmission to the model. Therefore, this metric can reduce the number of transmissions while ensuring that many data clumps are fixed. 

\subsubsection{Update time}

The time or frequency of when a particular file is updated can also be a metric to ascertain whether a refactoring is warranted. These information can be obtained from \ac{VCS} like Git as they are part of commits.

Files that have been recently updated can have new data clumps that should be refactored. These data clumps can be refactored more easily because it is more improbable that they have manifested so that other parts of the project or external projects depend on the existence of the data clump. 

On the other hand, files that were untouched for longer times might also contain data clumps that exist but are not refactored. One reason can be that developers are not familiar with these source files and are afraid to change them. Depending on the concrete type of data clumps, they can nevertheless be refactored by automatic tools so that developers do not need to read this unknown source code.


\subsubsection{Combining metrics}

In many cases, one metric is not sufficient to evaluate the relevance of a data clump. Combining multiple metric is frequent tool to guarantee a more balanced view on code smells by weighting the importance of multiple metrics. The score of each metric is weighted by a constant factor provided in the configuration file, and the weighted average is calculated. 

 In order for weighting to work, the metrics must be scaled uniformly. If one metric allows values from 0 to one million, and the other a value from 0 to 1, the former is more likely to outweigh the latter regardless of the weighting factor. 
 
 Therefore, the metric combiner employs a two-step approach. In the first step and for each metric, the minimum and maximum value is calculated. Then, each metric value is scaled using the extrema resulting in a scaled metric ranging from zero to one.  

\subsection{Filtering}
In the filtering approach, a binary condition is tested on each item individually. if the item does not meet the criterion, it is removed from further consideration. hence, each item is analyzed independently. Theoretically, this could mean that all data clumps are filtered out if the filtering conditions are too lax and apply to all data clumps. 

These filters can be combined by logical operation (and, or), so that more complex filtering rules are possible. 

The following subsections presents some  criteria for filtering data clumps
\subsubsection{Ignore abstract methods}

Abstract methods or non-implemented methods in interfaces only describe a contract without functionality which must be implemented by derived classes. Therefore, the signature of such operations is essential because it describes the contract which must be obliged to by derived classes.

Hence, changing the signature of abstract methods is often avoided as derived classes or classes implementing the interface must be modified too. These classes may reside outside the scope of the abstract class or interface. Therefore, refactoring the signature of abstract methods has a higher chance of errors because the derived classes may not be known or cannot be changed. 

Hence, it can be useful to exclude abstract methods for purpose of data clump refactoring to prevent these issues. 

\subsubsection{Ignore specific identifiers}

In many programming languages, convention have been established so that certain identifiers have a well-known meaning. For instance, the \textit{serialVersionUID} attribute in Java indicates a version of a class that should be changed if new fields are added in order to ease serialization. Because this field name is fixed, it is more probable to establish a data clump although the developer has few choices to prevent this data clump.

To give another example, a class name ending with \enquote{Builder} indicates that the class builds or constructs something. Often this is related to another class (i.~e. the builder class constructs another class. This also has the potential to lead to data clumps because the builder class and the other class must have similar fields or parameters. 

In all these cases, the chance of getting false positive data clumps is higher. Therefore, filtering for these specific scenarios can be useful to receive only data clumps that are worth to refactor. 

\subsubsection{Ignoring generic types}
Generic types are types of which scarce information is available because the true type is only available at the run time. For instance, a generic list can be instantiated by providing a concrete type.

Because of this lack of information, generic types are often named using  single-letter identifiers (e.~g. \enquote{T}). hence, the name of the type does not confer much semantic meaning and is not helpful for finding a suitable class name. 

Additionally, because of the short identifiers, the risk of creating a possible data clump becomes greater as there is a higher chance of name collision.

Hence, filtering these data clumps can be one strategy to focus on important data clumps. 

\subsubsection{Ignore data clumps with specific annotations}

Annotations are markers attached to a method, field, parameter etc. that convey additional functionality to the respective component. For instance, an annotation can provide the \ac{JSON} key name which a field is serialized to and de-serialized from.  They exists in many programming languages, but the concrete syntax and applications differ.

With regard to data clumps, annotation can pose challenges as they often need to be moved to the extracted class as well. This does not always work as some annotation are only compatible with specific types of components (e.~g. parameters). Even if they are compatible with fields of an extracted class, further changes is often necessary to make the software working again. For instance, an annotation might indicate the source of data passed to a method, and this source must be available to the extracted class too. 

\subsection{Combining metrics and filters}

The differentiation between filters and metrics can be relaxed by using metrics as a filters, and filters as metrics thereby enabling the construction of complex filtering methods. 

In the first case, filters employ the metrics described in section \ref{sec:metrics}, a comparison operator, and a threshold value. If the metric applied to a data clump compared with the threshold value using the comparison function returns true, the data clump remains, otherwise it is filtered out. For instance, the metric can be the data clump size, comparison function can be the greater-than function, and the threshold value can be three. In this example only data clumps with more than three variables will remain for further consideration. 

Vice versa, a filter can be used as a metric. If a filter returns true on a specific item, the metric returns a different value than it would if the filter returns false. As result, the binary classification by a filter is still employed, but data clumps that do not fulfill the criteria of the filter still have a chance not be filtered out. 


\section{Configuration}\label{sec:config}
An important aspect for the usability of the tool is the possibility to configure the tool for the user's need. 

Since the goal of the tool is to allow the combination of multiple services and other tools in order to find and refactor data clumps, the user must be able to define which handler deals with which step (see section \ref{sec:pipeline}).

The configuration is provided by a \ac{JSON} file whose location needs to be provided to the tool via a command line argument. It might be argued that providing the configuration directly via the command line is better suited than a separate configuration file because they do not require the creations of files and are easier for users who start the tool just once. Nevertheless, configuration files are persistent and especially \ac{JSON} can be more easily structured so that they are easier to understand. As a result, only \ac{JSON} files will be used  for the configuration. 


Listing \ref{lst:config} shows an example configuration file:
  \begin{figure} [htbp!]
			\lstinputlisting
			[caption={ Example configuration file},
			label={lst:config},
			captionpos=b, basicstyle=\footnotesize, tabsize=2, showstringspaces=false,  numbers=left]
			{figures/chapter4/config.json}
		\end{figure}


In the beginning, the programming language is defined (l.~1). Then, the user can specify which handler should handle which step (l.~3-26). In this example, there are two steps. 

The first step (l.~4-9) establishes the project location and therefore the code obtaining context.

The second step (l.~ 10-26), deals with the detection of data clumps. Here a handler is used that allows detection and refactoring of data clumps via ChatGPT. This specific handler requires sub-handlers that perform intermediate task. For instance, the \textit{SimpleInstructionHandler} loads an instruction from a file path. The \textit{PairOfFileContent} handler submits pairs of file contents to ChatGPT.

It should be noted that the order of the handlers in the configuration  does not matter because the execution order is constant and in most cases, each step requires the context of a previous step so that parallel execution or vice-versa execution of steps is not possible. Only in the case of usage finding and name finding would a parallel execution make sense because none of these steps is dependent on the other.  However, this exception is not implemented.

Not all relevant objects are part of the pipeline. Some are outside of the pipeline and can be referenced by all handlers. Here, the large language model \ac{API} is initialized with ChatGPT(l.~ 28).

Each object that appears in the \ac{JSON} is instantiated using dependency injection. This means that the objects initially are registered at a central location but not fully instantiated until they are really needed. The general approach is as follows:
\begin{enumerate}
    \item The main program loads the \ac{JSON} and iterates over all objects
    \item Each object has a category (e.~g. \textit{LanguageModelInterface} (see l.~ 28)) and a name (e.~g. ChatGPTInterface). The name and its category might be identical if only one instance of the respective class is needed. 
    \item The name of the object and it parameters are registered at the central dependency manager under the category given
    \item If a handler or other object needs another object of a specific category, it can ask the central manager. If the respective object has  yet not been created, it will be created. Otherwise, the already instantiated object will be returned. 
\end{enumerate}
With this approach, a looser coupling can be achieved. The main program and the handlers does not need to know the exact details of the configuration but provide and receive them from a central dependency manager. Only the handler-specific configuration (e.~g. loading the project location  for the \textit{CodeObtaining} handler is still required and must be performed by each handler. These parameters are stored by the dependency manager. 
